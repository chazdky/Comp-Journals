\documentclass{report}

\usepackage{textcomp}
\usepackage[demo]{graphicx}
\usepackage{graphicx}
\usepackage{fancyhdr}
\usepackage{subcaption}
\usepackage{multicol}
\usepackage{outlines}
%===================================
\newcommand{\classinfo}{{\bf Research \\ Week 09}\\{\it CIT 288}\\{Chaz Davis}}
\newcommand{\semester}{BCTC \\ Spring 2020}
%===================================
\newcommand{\mysection}[1]{\section*{#1}}
\newcommand{\mysubsection}[2]{\textbf{\romannumeral #1) #2}}
%===================================
\setlength{\headheight}{15.2pt}
\pagestyle{fancy}
\fancyhf{}
\lhead{ \fancyplain{}{Chaz Davis} }
\rhead{ \fancyplain{}{\today} }
\cfoot{ \fancyplain{}{\thepage} }
\renewcommand{\headrulewidth}{0.5pt}
\renewcommand{\footrulewidth}{0pt}

%===================================
\title{\classinfo}
\author{\semester}
\date{\today}

%===================================

\begin{document}

\maketitle


%===================================
\mysection{\textbf{Part 1: Chapter 09 }}


\mysubsection{1}{Where would an application-level gateway exist physically in the network?  Specifically - outside the firewall?  Inside?  Investigate.}
It appears that application level gateways exist to be entered vi telnet or
ftp. Once you've gained access to the network. you can then specify the
applications to gain access to within the host network, or the computer/server
that you would like to access.

As for where in the network they exist, there are several configurations. The
first version of the gateway-level application setup has it setup on the
outside of the main firewall and router. The second configuration has a setup
inside the main firewall but in front of the main router. The third and
preferred variation has one inside the firewall and infront of the router, and
one outside the firewall, with this configuration, you can add additional
gateways in front of any subnet routers to protect access within those internal
networks as well. 


\noindent\mysubsection{2}{What is the difference in a ALG and a CLG gateway?  }
\\APG is more secure than circuit-level, ALG uses a unique program for each
applications, and CLG uses TCP connections, with know packet-filtering, ALG is
good for authentication and logging, and CLG grants access by port address. ALG
in not always transparent to users, CLG has no application level checking, ALG
is used for email, ftp, telnet, www. CLG can understand what is carried in the
packet.


\noindent\mysubsection{3}{What is a SOCKS server?  }
\\It is a general purpose proxy server that establishes a TCP connection to
another server on behalf of a client, then routes all the traffic back and
forth between the client and server. It works for any kind of network protocol
on any port. SOCKS V5 adds additional support for security and UDP.


\noindent\mysubsection{4}{What is a DMZ network and what types of systems would you expect to find on such networks?}
\\A DMZ network functions as  asubnetwork containing an organization's exposed,
outward-facing services. It acts as the exposed point to an untrusted network,
comonly the Internet.

You would see it commonly used with web servers, mail servers, and ftp servers.


\noindent\mysubsection{5}{How does an IPS differ from a firewall?}
\\An IPS, or intrusion prevention system, works with the firewall. It typically
sits between the outside world and the firewall. IPS proactively denies nework
traffic based on a security profile. If that packet represents a known security
threat.


\noindent\mysubsection{6}{How can an IPS attempt to block malicious activity?}
\\It uses the security profile configured by the sys-admin or cyber security
liason.


\noindent\mysubsection{7}{What information is used by a typical packet filtering firewall?}
\begin{itemize}
  \item{Source IP Address} the IP Address of the system that originated the IP
    packet.
  \item{Destination IP Address} The IP Address of the system the IP packet is
    trying to reach,
  \item{Source and destination Transport-Level Address} The transport level
    (eg. TCP or UDP) port number, which defines applications such as SNMP or
    TELNET. 
  \item{IP Protocol field} Defines the transport protocol
  \item{Interface} For a router with three or more ports, which interface of
    the router the packet came from or which interface of the router the packet
    is destined for.
\end{itemize}


\noindent\mysubsection{8}{What are some weaknesses of a packet filtering firewall?}
\begin{itemize}
  \item{They can be complex to configure} 
  \item{They cannot prevent application-layer attacks} 
  \item{They are susceptible to certain types of TCP/IP protocol attacks} 
  \item{The don not support user authentication of connections} 
  \item{the have limited logging capabilities} 
\end{itemize}


\noindent\mysubsection{9}{What are the common characteristics of a bastion host?}
\begin{itemize}
  \item{The bastion host hardware platform executes a secure os} 
  \item{Only the services that the network admin considers esential are
    installed on the bastion host} 
  \item{The bastion host may req. authentication before a user is allowed
    access to the proxy services} 
  \item{Each proxy is configured to allow access to only soecific host systems} 
  \item{Each proxy logs all traffic, each connection and its duration} 
  \item{Each proxy is independent of the other proxies on the bastion host and
    runs in a private secured directory} 
\end{itemize}


\noindent\mysubsection{10}{Why is it useful to have host-based firewalls?}
\\A host-based firewall is a software module used to secure an individual host.
\begin{itemize}
  \item{Filtering rules can be tailored to the host environment} 
  \item{Protection is provided independent of topology} 
  \item{Provides an additional layer of protection} 
\end{itemize}



%===================================

\end{document}
