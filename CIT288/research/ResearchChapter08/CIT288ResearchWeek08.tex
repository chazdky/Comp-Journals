\documentclass{report}

\usepackage{textcomp}
\usepackage[demo]{graphicx}
\usepackage{graphicx}
\usepackage{fancyhdr}
\usepackage{subcaption}
\usepackage{multicol}
\usepackage{outlines}
%===================================
\newcommand{\classinfo}{{\bf Research \\ Chapter 08}\\{\it CIT 288}\\{Chaz Davis}}
\newcommand{\semester}{BCTC \\ Spring 2020}
%===================================
\newcommand{\mysection}[1]{\section*{#1}}
\newcommand{\mysubsection}[2]{\textbf{\romannumeral #1) #2}}
%===================================
\setlength{\headheight}{15.2pt}
\pagestyle{fancy}
\fancyhf{}
\lhead{ \fancyplain{}{Chaz Davis} }
\rhead{ \fancyplain{}{\today} }
\cfoot{ \fancyplain{}{\thepage} }
\renewcommand{\headrulewidth}{0.5pt}
\renewcommand{\footrulewidth}{0pt}

%===================================
\title{\classinfo}
\author{\semester}
\date{\today}

%===================================

\begin{document}

\maketitle

%===================================
\mysection{\textbf{Chapter 08}}

\mysubsection{1}{List and briefly define four classes of intruders.}

\begin{itemize}
  \item{Cyber criminals:}  Are either individuals or members of an organized crime group with a 
    goal of financial reward.
  \item{ Activists:} Are either individuals, usually working as insiders, or members of a larger 
    group of outsider attackers, who are motivated by social or political causes.
  \item{ State-sponsored organizations:} Are groups of hackers sponsored by governments to conduct 
    espionage or sabotage activities.
  \item{Others:} Are hackers with motivations other than those listed above, including classic 
    hackers or crackers who are motivated by technical challenge or by peer-group esteem and 
    reputation.Many of those responsible for discovering new categories of buffer overflow 
    vulnerabilities could be regarded as members of this class
\end{itemize}


\noindent\mysubsection{2}{List and briefly describe the steps typically used by intruders when attacking a system.}
\begin{itemize}
  \item{1} Target acquisition and info gathering
  \item{2} Initial Access
  \item{3} Privilege escalation
  \item{4} Information gathering and/or system exploit
  \item{5} Maintaining access
  \item{6} Covering tracks
\end{itemize}

\noindent\mysubsection{3}{Describe the three logical components of an IDS.}
\begin{itemize}
  \item{Target Acquisition and info gathering} using scam websites to gain
    access to valuable information.
  \item{Initial Access} when the hacker succeeds in guessing passwords
    and other valuable information
  \item{Privilege Escalation} When the attackers have the control
    to adjust privileges
  \item{Information Gathering and/or System Exploit} when the attackers
    have the ability to modify files inside a target's system
  \item{Maintaining Access} when the attackers plant a backdoor or other
    malware to assure that they have access over time
  \item{Covering Tracks} editing log files so the target won't notice any
    changes
\end{itemize}

\noindent\mysubsection{4}{Describe the differences between a host-based IDS and a network-based IDS. How can their advantages be combined into a single system?}
\begin{itemize}
  \item{Sensor} it has responsibility in collecting data; input includes
    network packets, log files, system call traces.
  \item{Analyzer} receiving input from one or more sensors, responsible for
    determining if an intrusion has occurred. Te output of this component is
    an indication that an intrusion has occurred and may include evidence
    supporting the conclusion that an intrusion has occurred.
  \item{User Interface} it enables user to view the output of the system, or
    control the system behavior.
\end{itemize}

\noindent\mysubsection{5}{Explain the base-rate fallacy.}
\begin{itemize}
  \item{Host based IDS} monitors the characteristics of a single host and the
    events occurring within that host for suspicious activity.
  \item{Network-Based IDS} Monitors network traffic for a particular network
    segments and analyzes network, transport, and application protocols to
    identify suspicious activity.
\end{itemize}


\noindent\mysubsection{6}{What is the difference between anomaly detection and signature or heuristic intrusion detection?}
\begin{itemize}
  \item{Anomaly Detection} Involves the collextion of data relating to the
    behavior of legitamate users over a period of time. Then, current observed
    behavior is analyzed to determine with a high level of confidence whether
    this behavior is that of legitamate user or alternatively that of an
    intruder.
  \item{Signature or Heuristic detection} Uses a set of know malicious data
    patterns (signatures) or attack rules (heuristics) that are compared with
    current data behavior to decide if it is that of an intruder. it is also
    known as misuse detection. This approach can only identify known attacks
    for which it has patterns or rules.
\end{itemize}


\noindent\mysubsection{7}{What is the difference between signature detection and rule-based heuristic identification?}


In essence, anomaly approaches aim to define normal, or expected, behavior,
in order to identify malicious or unauthorized behavior. They can quickly and
efficiently identify known attacks. However, only anomaly detection is able to
detect unknown, zero-day attacks, as it starts with known good behavior and
identifies anoalies to it. Given this advantage, clearly anomaly detection
would be the proferred approach, were it not for the difficulty in collecting
and analying the data required, and the high level of false alarms.


\noindent\mysubsection{8}{What advantages do a Distributed HIDS provide over a single system HIDS?}


Traditionally, work on host-based IDSs focused on single-system stand alone
operaton. The typical organization, however, needs to defend a distributed
collection of hosts supported by a LAN or internetwork. Although it is possible
to mount a defense by using stand-alone IDSs on each host, a more effective
defense can be acheived by coordination and cooperation among IDSs across the
network.


\noindent\mysubsection{9}{What are possible locations for NIDS sensors?}
\begin{itemize}
  \item{Between the external Firewall and the internet} 
  \item{Just inside the external firwall} 
  \item{Inside internal Firewalls} 
    \subitem{-}{between desktops/networks and internal firewall} 
    \subitem{-}{between internal firewalls and servers and data resources}  
\end{itemize}


\noindent\mysubsection{10}{What is a honeypot?}
\\Honeypots are resources that have no production value. There is no legitimate
reason for anyone outside the network to interact with a honeypot. Thus, any
attempt to communicate with the system is most likely a probe, scan, or attack.
Conversely, if a honeypot initiates outbound communication, the system has
probably been compromised.



%===================================

\end{document}
