\documentclass[../CIT288SecurityResearchNotebook.tex]{subfiles}

\begin{document}

%%%%%%%%%%%%%%%%%%%%%%%%%%%%%%%%%%%%%%%%%%%%%%%%%%%%%
%%%%%%%%%%%%%%%%%%%%%%%%%%%%%%%%%%%%%%%%%%%%%%%%%%%%%

\chapter[Research Ch. 04]{Research\linebreak[1] Chapter 04 \hspace*{\fill}{\date}}
\noindent\textbf{{Networking Security} \hspace*{\fill}{\textbf{CIT 288}}}\linebreak[1]
{{Spring 2020} \hspace*{\fill}{Chaz Davis}}                             
%===================================
%===================================

\mysection{What is a Trust Framework?  Give at least one real-world example of its use in Microsoft or Linux offerings}

In digital identity systems, a trust framework functions as a certification 
program. It enables a party who accepts a digital identity credential (called 
the relying party) to trust the identity, security, and privacy policies of 
the party who issues the credential (called the identity service provider) 
and vice versa.

A trust framework is primarily a legal framework that captures a set 
of activities and responsibilities of participatin entities in a 
way that it promotes trust among those entities. Trust framework 
may or may not be accompanied by technical spec.

\mysection{Who is OIX?  ICF? ICAM? OITF? OIDF?  
\\What is the mission of each of these?}

\begin{itemize}\centering
	\item{OIDF} The OpenID Foundation is an international nonprofit 
		organization of individuals and companies committed to enabling, 
		promoting, and protecting OpenID technologies. OIDF assists the 
		community by providing needed infra- structure and help in promoting 
		and supporting expanded adoption of OpenID.
	\item{ICF} The Information Card Foundation is a nonprofit community of 
		companies and individuals working together to evolve the 
		Information Card ecosystem. Information Cards are personal 
		digital identities people can use online, and the key 
		component of identity metasystems.  Visually, each 
		Information Card has a card-shaped picture and a card 
		name associated with it that enable people to organize their digital 
		identities and to easily select one they want to use for any given 
		interaction.
	\item{OITF} The Open Identity Trust Framework is a standardized,
		open specification of a trust framework for identity and attribute 
		exchange, developed jointly by OIDF and ICF.
	\item{OIX} The Open Identity Exchange Corporation is an independent, 
		neutral, international provider of certification trust frameworks 
		conforming to the Open Identity Trust Frameworks model.
	\item{AXN} An Attribute Exchange Network (AXN) is an online 
		Internet-scale gateway for identity service providers 
		and relying parties to efficiently access user-asserted, 
		permissioned, and verified online identity attributes in 
		high volumes at affordable costs
\end{itemize}

\mysection{Explain what an AXN does.}

\begin{itemize}\centering
	\item{Subjects} These are users of an RP’s services, 
		including customers, employees, trading partners, and subscribers.
	\item{Attribute providers (APs)} APs are entities acknowledged by 
		the community of interest as being able to verify given 
		attributes as presented by subjects and which are equipped 
		through the AXN to create conformant attribute 
		credentials according to the rules and agreements of the AXN. 
		Some APs will be sources of authority for certain information; 
		more commonly APs will be brokers of derived attributes.
	\item{Identity providers (IDPs)} These are entities able to 
		authenticate user credentials and to vouch for the names 
		(or pseudonyms or handles) of subjects, and which are 
		equipped through the AXN or some other compatible 
		Identity and Access Management (IDAM) system to create 
		digital identities that may be used to index user 
		attributes.
\end{itemize}
There are also the following important support elements as part on an AXN:
\begin{itemize}\centering
	\item{Assessors} Assessors evaluate identity service providers 
		and RPs and certify that they are capable of following 
		the OITF provider’s blueprint.
	\item{Auditors} These entities may be called on to check that 
		parties’ practices have been in line with what was agreed 
		for the OITF.
	\item{Dispute resolvers} These entities provide arbitration 
		and dispute resolution under OIX guidelines.
	\item{Trust framework providers} A trust framework provider 
		is an organization that translates the requirements 
		of policymakers into an own blueprint for a trust framework 
		that it then proceeds to build, doing so in a way that 
		is consistent with the minimum requirements set 
		out in the OITF specification. In almost all cases, 
		there will be a reasonably obvious candidate organization 
		to take on this role, for each industry sector or large 
		organization that decides it is appropriate to interoperate 
		with an AXN.
\end{itemize}

\mysection{How are capability tickets utilized?}


{\bf{capability ticket}} A discretionary access control technique 
organized by subject. For each subject, the capability 
ticket lists objects and their permitted access 
rights by this subject.

When it is desired to determine which subjects have which access 
rights to a particular resource, ACLs are convenient, 
because each ACL provides the information for a given resource. 
However, this data structure is not convenient for determining 
the access rights available to a specific user.


Decomposition by rows yields capability tickets. A capability 
ticket specifies authorized objects and operations for a 
particular user. Each user has a number of tickets and 
may be authorized to loan or give them to others. 
Because tickets may be dispersed around the system, 
they present a greater security problem than access 
control lists. The integrity of the ticket must be 
protected, and guaranteed (usually by the operating system).
In particular, the ticket must be unforgeable. 
One way to accomplish this is to have the operating system 
hold all tickets on behalf of users. These tickets would have to 
be held in a region of memory inaccessible to users. 
Another alternative is to include an unforgeable 
token in the capability. This could be a large random 
password, or a cryptographic message authentication code. 
This value is verified by the relevant resource whenever 
access is requested. This form of capability ticket is 
appropriate for use in a distributed environment, 
when the security of its contents cannot be guaranteed.


The convenient and inconvenient aspects of capability 
tickets are the opposite of those for ACLs. 
It is easy to determine the set of access rights that a 
given user has, but more difficult to determine the 
list of users with specific access rights for a specific resource.

\mysection{Briefly define the difference between DAC and MAC.}


{\bf{Discretionary Access Control}}
\begin{outline}\centering
	\1 the control access is defined based on the requestor identity 
	and the access rule authorizations.
	\1 It permits the requestors only to perform the allowed activity
	\1 The above policy is termed as a discretionary
	\1 Because the entity must have access rights to permit another
	entity by its own decision
	\1 It enables another entity to access some resource
\end{outline}

\clearpage

{\bf{Mandatory Access Control}}
\begin{outline}\centering
	\1 The control access is defined based on comparing the security
	labels with the security clearances.
	\1 The security label indicates whether the system resource
	is sensitive or critical.
	\1 The security clearance refers the system entities which are
	eligible to access the certain resources.
	\1 The above policy is termed as mandatory
	\1 Because the entity may or may not have eligibility to access
	the resource by its own decision and enable another entity
	to access some resource.
\end{outline}


\mysection{How does RBAC relate to DAC and MAC?}

RBAC is a Role Based Access Control. The user roles are defined 
within the system and the access allowed based on the given roles.


All three are not mutually exclusive and the access control mechanism
can employ two or all the three of the policies to cover different 
types of system resource.


The RBAC may use the discretionary or the 
mandatory mechanism for user roles.

\mysection{List and define the three classes of subject 
\\in an access control system.}

{\bf{Access Control System}} The access control system is embodied in the authorization database. It states the types of permitted access, circumstances for the permission and who are all permitted. 
\linebreak[1]\vspace{0.5cm}

\begin{outline}\centering
- The access control system defines 
the three classes of the subject with different access rights:
	\1 Owner
	\1 Group
	\1 World
\end{outline}


\mysection{What is the difference between an access control list and a capability ticket?}

Access Control Lists can be simply explained as the mechanism that allows the
permission on who  can access the object. Capability Ticket refers to the process that
shows what objects are allowed to access and what operations are allowed on it.

\mysection{In the NIST RBAC model, what is the difference between SSD and DSD?}

SSD is a constraint of the National Institute of Standards and Technology role based
access(NIST RBAC) model. It enables a set of mutually exclusive roles. If one role is
assigned to a user from a set, then the user may not be assigned to any other roles
from that set.
DSD is a constraint of NIST RBAC model. DSD relation is used to limit the permissions
available to the user. DSD places constraints on the role which is activated within or
across a user's session to limit available permissions.

\mysection{What is a protection domain?}

A protection domain is a grouping of code source and permissions. A protection domain
represents all the permissions that are granted to a particular code source. In the
default implementation of the Policy class, a protection domain is one grant entry in
the file.

\mysection{UNIX treats file directories in the same fashion as files; that is, both are defined by the same type of data structure, called an inode. As with files, directories include a nine-bit protection string. If care is not taken, this can create access control problems. For example, consider a file with protection mode 644 (octal) contained in a directory with protection mode 730. How might the file be compromised in this case?}

If the file's octal code is 644 then that represents
\begin{outline}
	\centering
	\1 read and write access for the owner
	\1 read access to the group
	\1 read access to the everyone else
\end{outline}

If the directory's octal code is 730 then that represents
\begin{outline}
	\centering
	\1 read and write and execute access for the owner
	\1 write and execute access to the group
	\1 and null value access for all other users
\end{outline}

Since the file has only read permission for the group and others, the file has no write
and execute permissions for the group and others. Whereas the directory has the write
and execute permissions for the users group. So, the member of the group may change the
content of the file or the file may be deleted. Thus, the permissions given to the file
are of no use.
The file has read permission for everyone else whereas the directory has no permission
for others. Thus, the content of the file cannot be read by the others. The permissions
given to the file are of no use.

\mysection{What is a session?}

A session is a temporary and interactive information exchange between two or more
communicating devices. A browser may be making a series of http requests and
transactions all initiated by the same user. Typically a session is started when a user
authenticates their identity using a password or another authentication protocol.

\mysection{Give a brief overview of credential management.}

Credential Management is the set of practices that an organization uses to issue, track,
update, and revoke credentials for identities within their context.

\end{document}
