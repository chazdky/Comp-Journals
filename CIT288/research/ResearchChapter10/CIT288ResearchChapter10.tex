\documentclass{report}
\usepackage[english]{babel}


\usepackage{textcomp}
\usepackage[demo]{graphicx}
\usepackage{graphicx}
\usepackage{fancyhdr}
\usepackage{minted}
\usepackage{subcaption}
\usepackage{multicol}
\usepackage{outlines}
%===================================
\newcommand{\classinfo}{{\bf Research \\ Week 10}\\{\it CIT 288}\\{Chaz Davis}}
\newcommand{\semester}{BCTC \\ Spring 2020}
%===================================
\newcommand{\mysection}[1]{\section*{#1}}
\newcommand{\mysubsection}[2]{\textbf{\romannumeral #1) #2}}
%===================================
\setlength{\headheight}{15.2pt}
\pagestyle{fancy}
\fancyhf{}
\lhead{ \fancyplain{}{Chaz Davis} }
\rhead{ \fancyplain{}{\today} }
\cfoot{ \fancyplain{}{\thepage} }
\renewcommand{\headrulewidth}{0.5pt}
\renewcommand{\footrulewidth}{0pt}

%===================================
\title{\classinfo}
\author{\semester}
\date{\today}

%===================================

\begin{document}

\maketitle

%===================================
\mysection{\textbf{Chapter 10}}

\mysubsection{1}{What does buffer overflow mean?}
\\A buffer overflow occurs when data written to a buffer also corrupts data
values in memory addresses adjacent to the destination buffer due to
insufficient bounds checking. This can occur when copying data from one buffer
to another without first checking that the data fits within the destination
buffer.

\noindent\mysubsection{2}{Are Windows systems the only vulnerable operating system?  }
\\No, I would say the notion that its the most vulnerable comes from the fact
that it overwhelmingly olds most of the home computing dektop os market. That
in itself makes it the largest target, therefore, the appearance of most
vulnerable. No one was really trying to attack macs and/or linux, because they
were less than two percent of home computing at one point. The largest
advantage linux has, is that its open source, all of the code is out there for
free, so anyone can see view it, find holes, and patch holes.

\noindent\mysubsection{3}{Can you find an outside example of a non-windows overflow attack?}
\\Yes, Linux, due to being C and C++ based, is the most susceptible system to
buffer overflows. Simple C programs can be used to create buffer overflows.
First you would want to allocate a slot of memory, and then you would want to
pass arguments that it doesnt precheck. this is when Python programming
becomes very, very handy. Say i create a program in c and allocate 500 bytes
of memory to hold the variable. call it something like vuln.c

\begin{minted}
[
frame=lines,
framesep=2mm,
baselinestretch=1.2,
fontsize=\footnotesize,
linenos
]
{c}
#include <stdio.h>
#include <string.h>

int main (int argc, char** argv)
{
    char buffer[500];
    strcpy(buffer, argv[1]);

    return 0;
}

\end{minted}

Now, you could simply pass "Hello" to the command line and all would work well,
bu to create a simple overflow with a segfault error you could run something
like {\scriptsize{\verb$python -c 'print "\x41" * 508'$}\normalsize} 
and it would save the `a' character into memory 508 times, then we would also
over write the entirety of the return address. This would have the return
address as {\scriptsize{\verb$0x414141$}\normalsize} . So, we could set a virus or exploit at that point of address
memory like this in assembler where the system interupt and gives
you access to a root shell.

\noindent\mysubsection{4}{What kind of locations do buffer overflows typically attack in a process address space?}
\\It would attack the Base Pointer for the return space of the memory. they
would target the stack, the heap, or the data section to do this.

\noindent\mysubsection{5}{What programming languages are susceptible to buffer overflow?}
\\Compiled and memory managed languages, without bounds. SO, C, Cpp, or
Assembler are the best, they provide direct access to root and low-level
memory. Although, you can adapt just about any language to be exploited with
the correct code and implementations.

\noindent\mysubsection{6}{What does an off-by-one attack do to a system?}
\\Buffer Overflow: Off-by-One. The program writes just past the 
bounds of allocated memory, which could corrupt data, crash the 
program, or lead to the execution of malicious code. ... The result 
is that information on the call stack is overwritten, including the 
function's return pointer.

\noindent\mysubsection{7}{What is a Guard page?  How effective is this defense in your opinion -- no wrong answer here..}
\\Guard pages are unmapped pages placed between all memory 
allocations of one page or larger. The guard page causes a 
segmentation fault upon any access. Thus useful in implementing 
protection for areas such as network interfacing, virtual machines, 
and interpreters

Guard pages have a high degree of overhead because they fragment 
the kernel's memory map and can increase the amount of virtual 
space considerably. Their effectiveness depends on the size and 
pattern of allocations; they are often more effective as a 
debugging facility than an operational security measure.

\noindent\mysubsection{8}{Shellcode has to be "position independent".  What does this mean?}
\\ Shellcode is the machine code that is injected into the flow of 
a program as the result of an exploit. It generally must be 
position independent as you can't usually control where it will be 
loaded in memory

\noindent\mysubsection{9}{What are the possible consequences of a buffer overflow occurring?}
\\The user would gain a login shell with root privileges ad be able to access
all resources and all memory access.

\noindent\mysubsection{10}{What are the two key elements that must be identified in order to implement a buffer overflow?}
\begin{itemize}
  \item{To identify a buffer vulnerability in some program that can be
    triggered using externally sourced data ubder the attackers control} 
  \item{To understand how that buffer will be stored in the process memory, and
    hence the potential for corrupting adjacent memory locations and
  potentially altering the flow of execution of the program} 
\end{itemize}

\noindent\mysubsection{11}{List some of the different operations an attacker may design shell code to perform.}
\begin{itemize}
  \item{Creating a remote shell to listen to services performed on the user's system. } 
  \item{Creating a reverse shell to return control of the user's system. } 
  \item{Create a shell on the user system by using local exploits in the program. } 
  \item{Break out of the firewall rules in the system to avoid attack prevention.} 
\end{itemize}

\noindent\mysubsection{12}{List and briefly describe some of the defenses against buffer overflows that can be used when compiling new programs.}
\\Developing a program with modern high-level programming language. 
Using safe code techniques like buffer validation and safe functions. 
Using safe library to implement the program. 
Using some stack protection mechanism

\noindent\mysubsection{13}{Describe how a global data area overflow attack is implemented.}
\\The buffer which is targeted is located at the global data area of a program. 
The buffer which is used as global has been overwritten by using unsafe buffer operations; 
it changes the pointer memory locations used in the buffer. 
The program which is attacked will make a call to overwritten function; 
the function will send the program's control to the shellcode which is written by the attacker.

\noindent\mysubsection{14}{Do you think Linux or Microsoft web servers are the most vulnerable?  This is open ended.  Don't get too lengthy - but look into it a bit and give a response.  }
\\I believe that Linux is the more secure option overall. I must say that based
on bas-configuration out-of-the-box, windows is probably a touch more secure.
But as you decide to work on microsoft servers, adding additional programs,
users, and the like, thats when microsoft servers get weaker.
Those are also the things that make Linux Stronger. The more you tinker in the
correct ways, the more secure and less vulnerable they will be.
While both are meant to work on an abundance of hardwares, i believe that linux
works better on alder out of date architectures than windows. Windows at
somepoint puts an end of life on their software. Linux doesnt, its all
available out there, so if I wanted to fire up a PC and run the second linux
kernel version. its there for me to do so, and chances are more tinkerers are
out there anbd theyve found veulnerabilities and ways to patch those holes. and
ways to make it run on things it wasnt intended to run on. You have the ability
to modify and write drivers for any device you want to move things to, as well.


%===================================

\end{document}
