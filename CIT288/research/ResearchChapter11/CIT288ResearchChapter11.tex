\documentclass{report}

\usepackage{textcomp}
\usepackage[demo]{graphicx}
\usepackage{graphicx}
\usepackage{fancyhdr}
\usepackage{subcaption}
\usepackage{multicol}
\usepackage{outlines}
%===================================
\newcommand{\classinfo}{{\bf Research \\ Week 11}\\{\it CIT 288}\\{Chaz Davis}}
\newcommand{\semester}{BCTC \\ Spring 2020}
%===================================
\newcommand{\mysection}[1]{\section*{#1}}
\newcommand{\mysubsection}[2]{\textbf{\romannumeral #1) #2}}
%===================================
\setlength{\headheight}{15.2pt}
\pagestyle{fancy}
\fancyhf{}
\lhead{ \fancyplain{}{Chaz Davis} }
\rhead{ \fancyplain{}{\today} }
\cfoot{ \fancyplain{}{\thepage} }
\renewcommand{\headrulewidth}{0.5pt}
\renewcommand{\footrulewidth}{0pt}

%===================================
\title{\classinfo}
\author{\semester}
\date{\today}

%===================================

\begin{document}

\maketitle

%===================================
\mysection{\textbf{Chapter 11}}

\mysubsection{1}{Define defensive programming.}
\\Defensive programming is when a programmer anticipates problems and writes code to deal with them.
That said, the whole point of defensive programming is guarding against errors you don’t expect. 
A defensive programmer is on the watch for trouble, to avoid it before it can cause real problems. 
The idea is not to write code that never fails. 
That is a utopian dream. The idea to make the code fail beautifully in case of any unexpected issue. 
Fail beautifully can mean any one of the following.
\begin{itemize}
  \item{Fail Early} 
  \item{Fail Safe} 
  \item{Fail Clearly} 
\end{itemize}

\noindent\mysubsection{2}{Define input fuzzing. State where this technique should be used.}
\\FUZZ TESTING (fuzzing) is a software testing technique that inputs invalid or 
random data called FUZZ into the software system to discover 
coding errors and security loopholes. Data is inputted using 
automated or semi-automated testing techniques after which the 
system is monitored for various exceptions, such as crashing down 
of the system or failing built-in code, etc.

Fuzz testing was originally developed by Barton Miller at the 
University of Wisconsin in 1989. Fuzz testing or fuzzing is a 
Software testing technique, and it is a type of Security Testing.

\noindent\mysubsection{3}{List several software security concerns associated writing safe program code.}
\begin{itemize}
  \item{Correct Algorithm Implementation} 
  \item{Ensuring that machine language corresponds to the algorithm} 
  \item{Correct interpretation of data values} 
  \item{Correct use of memory} 
  \item{Preventing race conditions with shared memory} 
\end{itemize}

\noindent\mysubsection{4}{Define the principle of least privilege.}
\\Programs should execute with the least amount of privilges needed to
complete their function.

\noindent\mysubsection{5}{Identify several issues associated with the correct creation and use of a temporary file in a shared directory.}
\\The temporary file must not be accessed by another process. An attacker
could guess the name of the temporary file and create it in between the time
the program checks if it exists and subsequently creating it. The program
could be redirected and would overwrite an existing file. Thus the use of
secure system calls is advised to avoid race conditions.

%===================================

\end{document}
