\documentclass[../CIT288SecurityResearchNotebook.tex]{subfiles}

\begin{document}

%%%%%%%%%%%%%%%%%%%%%%%%%%%%%%%%%%%%%%%%%%%%%%%%%%%%%
%%%%%%%%%%%%%%%%%%%%%%%%%%%%%%%%%%%%%%%%%%%%%%%%%%%%%

\chapter[Research Ch. 02]{Research\linebreak[1] Chapter 02 \hspace*{\fill}{\date}}
\noindent\textbf{{Network Security} \hspace*{\fill}{\textbf{CIT 288}}}\linebreak[1]
{{Spring 2020} \hspace*{\fill}{Chaz Davis}}                             
%===================================

\mysection{What are the essential ingredients of a symmetric cipher?}

\begin{itemize}
	\item plain text
	\item Encryption Algorithm
	\item Secret Key
	\item Cipher Text
	\item Decryption Algorithm
\end{itemize}

\mysection{How many keys are required for two people to communicate via a symmetric cipher?}

 1, the same key thats used to encrypt the message is used to decrypt the message

\mysection{What are two principal requirements for the secure use of symmetric encryption?}

\begin{itemize}
	\item A strong encryption algorithm. The opponent should be unable to decrypt ciphertext or discover the key even if he or
	she is in possession.
	\item Sender and receiver must have obtained copies of the secret key in a secure fashion and must keep the key secure
\end{itemize}

\mysection{List three approaches to message authentication}

\begin{itemize}
	\item Using conventional encryption
	\item Using Public-key encryption
	\item Using a secret value
\end{itemize}

\mysection{What is message authentication?}

small block of data, that is appended to a message to assure that the sender is authentic and that the message is
	unaltered.

\mysection{Briefly describe the three schemes found in figure 2.3}

\begin{itemize}
	\item A
	\item B
	\item C
\end{itemize}

\mysection{What properties must a hash function have to be useful for message authentication?}

\begin{itemize}
	\item Plain text
	\item Encryption Algorithm
	\item Public and Private keys
	\item Cipher text
\end{itemize}

\mysection{List and briefly describe three uses of a public-key cryptosystem}

\begin{itemize}
	\item Encryption/Decryption: The sender encrypts a message with the recipients public key.
	\item Digital Signature: The sender ``signs'' a message with its private key.
	\item Key Exchange: Two sides cooperate to exchange as a session key. Several different approaches are possible, involving
	the private keys of one or both parties.
\end{itemize}

\mysection{What is the difference between a private key and a secret key?}

The key used in conventional encryption is typically referred to as a secret key. the two keys used for public-key
	encryption are referred to as the public key and the private key.

\mysection{What is a digital signature?}

A mechanism for authenticating a message. Bob uses a secure hash function, sch as SHA-512, to generate a hash value
	for the message and then encrypts the hash code with his private key, creating a digital signature. Bob sends the
	message with the signature attached. When Alice receives the message she calculates a hash value for the message,
	decrypts the signature using Bob's public key and compares it to Bob's hash value. If the two hash values match,
	Alice is assured that the message must have been signed by Bob. It is important to emphasize that the digital
	signature does not provide confidentiality.

\mysection{What is public-key certificate?}

A certificate consists of a public-key plus a user ID of the key owner, with the whole block signed by a trusted
	third-party (= certificate authority CA) The user can then publish the certificate and anyone needing the users
	public-key can obtain the certificate and verify that it is valid by means of the attached signature.

\mysection{How can public-key encryption be used to distribute a secret key?}

\begin{itemize}
	\item Digital Envelope
	\item Prepare a message
	\item Generate a random symmetric key that will be used the time only
	\item Encrypt that message using symmetric key encryption with the one-time key. 
	\item Encrypt the one-time key using public-key encryption
\end{itemize}

\end{document}
