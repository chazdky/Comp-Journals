\documentclass{report}

\usepackage{textcomp}
\usepackage[demo]{graphicx}
\usepackage{graphicx}
\usepackage{fancyhdr}
\usepackage{subcaption}
\usepackage{multicol}
\usepackage{outlines}
%===================================
\newcommand{\classinfo}{{\bf RHEL 134 \\ Week 03 Labs}\\{\it CIT 218}\\{Chaz Davis}}
\newcommand{\semester}{BCTC \\ Spring 2020}
%===================================
\newcommand{\mysection}[1]{\section*{#1}}
\newcommand{\mysubsection}[2]{\textbf{\romannumeral #1) #2}}
%===================================
\setlength{\headheight}{15.2pt}
\pagestyle{fancy}
\fancyhf{}
\lhead{ \fancyplain{}{Chaz Davis} }
\rhead{ \fancyplain{}{\today} }
\cfoot{ \fancyplain{}{\thepage} }
\renewcommand{\headrulewidth}{0.5pt}
\renewcommand{\footrulewidth}{0pt}

%===================================
\title{\classinfo}
\author{\semester}
\date{\today}

%===================================

\begin{document}

\maketitle

%===================================
\mysection{\textbf{Part 1: Questions}}


\mysubsection{1}{Create a file named “firstname_lastname”. Set an ACL for your current user to have read-only access to the file. Provide the ACL details of the file.}\\
a


\noindent\mysubsection{2}{Create a directory named “CIT218” in your home directory. Set an ACL to allow recursive read/write access on the directory for the “student” group. Provide the ACL details of the directory.}\\
a


\noindent\mysubsection{3}{Change the mode of SELinux to disabled. Provide the output.}\\
a



\noindent\mysubsection{4}{Create a file named “firstname_lastname”. Change the file context to httpd_sys_content_t.  Provide the SELinux details of the file.}\\
a



\noindent\mysubsection{5}{Allow the HTTPS service through the firewall. Make it permanent and reload the firewall. }\\
a


%===================================

\end{document}
