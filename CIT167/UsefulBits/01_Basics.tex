\documentclass[../EngineeringJournal_CDavis.tex]{subfiles}

\begin{document}

%%%%%%%%%%%%%%%%%%%%%%%%%%%%%%%%%%%%%%%%%%%%%%%%%%%%%
%%%%%%%%%%%%%%%%%%%%%%%%%%%%%%%%%%%%%%%%%%%%%%%%%%%%%

\chapter*{Useful Bits: \linebreak[1]Basics \hspace*{\fill}{2020}}
\noindent\textbf{{Packet Tracer Labs} \hspace*{\fill}{\textbf{CIT 167}}}\linebreak[1]
{{Spring 2020} \hspace*{\fill}{Chaz Davis}}                             
\addcontentsline{toc}{chapter}{Useful Bits: Basics}

%===================================

\hspace{0.2cm}
\begin{tcolorbox}[width=6.3in]
\scriptsize 
\begin{verbatim}
User can interact with a shell with a CLI or a GUI.

Cisco IOS is used for cisco devices

The most common way to configure a HOME router is with a web broswer

Methods to access the CLI environment:
	-Console : Physical management port that provides out of band access, 
		  works even without connectivity
	-SSH : Secure CLI connection, requires active networking services
	-Telnet: Like SSH but insecure because everything is sent in cleartext

Some devices also support a legacy AUX port used to establish remote CLI session using a modem.
	It works similarly to console connection.

Some useful terminal emulators via SSH/Telnet:
	-PuTTY
	-Tera term
	-OS X terminal
	-SecureCRT

For security Cisco IOS separates management access into two modes:
	-User EXEC mode: Limited capabilities but useful for basic operations and monitoring,
			 Doesn't allow change to configurations
			 CLI prompt ends with a >
			 referred as 'view-only' mode
	
	-Privileged EXEC mode: Higher configuration modes can only be accessed this mode
			       CLI prompt ends with a #
				

To configure the device the user must enter Global configuration mode:
	- it is identified by a command prompt that ends with (config)#
	- changes here affect the device as a whole
	- from here you can access more specifi sub-configuration mode for specific function of the IOS device:
		Two common subs are:
			-Line configuration Mode: (config-line)#
			-Interface configuration Mode: (config-if)#


If you type 'enable' on command line you go from user EXEC to privileged EXEC mode
If you type 'disable' on command line you return to the user EXEC mode

From privileged EXEC mode if i type 'configure terminal' i go to Global configuration mode
To return to privileged EXEC mode type 'exit'
\end{verbatim}
\end{tcolorbox}
\hspace{0.2cm}
\normalsize  

\hspace{0.2cm}
\begin{tcolorbox}[width=6.3in]
\scriptsize 
\begin{verbatim}
From Global configuration mode if i type 'interface <interface-name> <interface-number' 
	  i get to the specific interface configuration
for the specified interface, the same for line 'line <line-name> <line-number>'.
	You don't need to return to global configuration mode to switch between sub-configurations.
To return to Global Configuration mode (from sub-configuration) type 'exit'.
To return directly to Privileged exec mode type 'end' or press 'Ctrl+Z'

The general command syntax is the command followed by keywords and arguments:
	-Keyword : a specific parameter defined in the operating system
	-Argument: not predefined, a variable or value user-defined.

Convention:
	boldface: commands and literals that you enter as shown
	italics: arguments for which you suplly values
	[x] : Optional element
	{x} : Required element
	[x {y | z}]: Required choice within an optional argument

IOS helps:
	-Context-sensitive help:
		Helps you find quickly which command are available in each command mode
		To access it type '?' at the CLI
		It can even be used to check what parameters a particular command accept 
			or to finish the name of a command.
	Command syntax check:
		If a command is entered wrong, the CLI provides the user a feedback about what is wrong.
\end{verbatim}
\end{tcolorbox}
\hspace{0.2cm}
\normalsize  


\hspace{0.2cm}
\begin{tcolorbox}[width=6.3in]
  \href{http://etherealmind.com/cisco-ios-cli-shortcuts}{Cisco IOS shortcuts} 
\scriptsize 
\begin{verbatim}
IMPORTANT SHORTCUTS : Ctrl-Shift-6 To abort a command mid-stream 
			  (Useful for mistyped commands and cisco IOS attempting translate it with DNS)
		      Ctrl+R refresh last command (example if an output of a interface down/up 
			    shows in the middle of typing a command)

Hostnames:
	-Case sensitive
	-to change hostname use the command 'hostname <hostname>' in Global Config Mode
	-to reset default hostname use the command 'no hostname' in Global Config Mode

Passwords:
	-use 'enable secret <password>' in Global Config Mode to set password for Privileged Exec Mode
	-to set a password in user EXEC mode, type 'password <password>' in line console configuration mode 
		('line console 0' global config command)

	 next enable user login with the 'login' command
	-to set a password for VTY (Virtual terminal) lines used for SSH and Telnet enter line VTY config mode 
		with 'line vty 0 15' (if 16 vty lines)
	 next set password with 'password <password>' and enable login with 'login'
	-To encrypt passwords:
		Use command 'service password-encryption', this applies only to configuration files

To check the config for encryption run 'show running-config'

To add a banner message of the day use 'banner motd # the message of the day # '

There are two types of config:
	-startup config ,stored in NVRAM, to view it use 'show startup-config'
	-running config ,stored in RAM
To save changes made to the running config in the startup config do 'copy running-config startup-config'
To restore the startup config run in privileged EXEC mode 'reload'
If unwanted changes were made to the startup config file it is 
      possible to remove it by using 'erase startup-config'
\end{verbatim}
\end{tcolorbox}
\hspace{0.2cm}
\normalsize  

\hspace{0.2cm}
\begin{tcolorbox}[width=6.3in]
\scriptsize 
\begin{verbatim}
Types of network media include twisted-pair copper cables, 
			      fiber-optic cables, 
			      coaxial cables, 
			      or wireless as shown in the figure. 

Difference between them:

    Distance the media can successfully carry a signal

    Environment in which the media is to be installed

    Amount of data and the speed at which it must be transmitted

    Cost of the media and installation
\end{verbatim}
\end{tcolorbox}
\hspace{0.2cm}
\normalsize  

\hspace{0.2cm}
\begin{tcolorbox}[width=6.3in]
\scriptsize 
\begin{verbatim}
Cisco IOS Layer 2 switches have physical ports for devices to connect. 
These ports do not support Layer 3 IP addresses. 
Therefore, switches have one or more switch virtual interfaces (SVIs)

IP address information can be added:
	-manually
	-using DHCP

To configure SVI use 'interface vlan 1' in global config mode.
Assign an ip address using 'ip address <ip-address> <subnet-mask>'
Enable the inteface using 'no shutdown'

To see a brief interface ip screen : 'show ip inteface brief'

To test the connectivity of a device on a network 
or of a website use the command 'ping <ip-address>'

\end{verbatim}
\end{tcolorbox}
\hspace{0.2cm}
\normalsize  

\newpage

%===================================

\end{document}
