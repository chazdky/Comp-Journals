\documentclass[../EngineeringJournal_CDavis.tex]{subfiles}

\begin{document}

%%%%%%%%%%%%%%%%%%%%%%%%%%%%%%%%%%%%%%%%%%%%%%%%%%%%%
%%%%%%%%%%%%%%%%%%%%%%%%%%%%%%%%%%%%%%%%%%%%%%%%%%%%%

\chapter*{SSH and Port \linebreak[1] Security \hspace*{\fill}{2020}}
\noindent\textbf{{Packet Tracer Labs} \hspace*{\fill}{\textbf{CIT 167}}}\linebreak[1]
{{Spring 2020} \hspace*{\fill}{Chaz Davis}}                             
\addcontentsline{toc}{chapter}{Useful Bits: SSH and Port Security}
%===================================

%===================================
\mysection{\textbf{Configuring SSH}}


\hspace{0.2cm}
\begin{tcolorbox}[width=6.3in]
\scriptsize 
\begin{outline}
  \1 All switch ports (interfaces) should be secured before the switch is deployed for production use.
      \2 Specify a single MAC address or a group of valid MAC addresses allowed on a port.
      \2 Specify that a port automatically shuts down if unauthorized MAC addresses are detected.
  \1 Sticky secure MAC addresses
      \2 MAC addresses that can be dynamically learned or manually configured, then 
      \2 stored in the address table and added to the running configuration by stick learning:
  \1 If the sticky secure MAC addresses are saved to the startup configuration file
      \2 then when the switch restarts or the interface shuts down, the interface does not need to relearn the addresses. 
      \2 If the sticky secure addresses are not saved, they will be lost.
  \1 If sticky learning is disabled by using the 
  {\verb$no switchport port-security mac-address sticky $}
  interface configuration mode command
      \2the sticky secure MAC addresses remain part of the address table, but are removed from the running configuration.
\end{outline}
  {\bf{Characteristics of sticky secure MAC addresses:}}
\begin{outline}
  \1 Learned dynamically, converted to sticky secure MAC addresses stored in the running-config.
  \1 Removed from the running-config if port security is disabled.
  \1 Lost when the switch reboots (power cycled).
  \1 Saving sticky secure MAC addresses in the startup-config makes them permanent and the switch retains them after a reboot.
  \1 Disabling sticky learning converts sticky MAC addresses to dynamic secure addresses and removes them from the running-config.
\end{outline}
\normalsize  
{\bf{Note:}} The port security feature will not work until port security is
enabled on the interface using the {\scriptsize{\verb$switchport port-security$}\normalsize}  command.
\end{tcolorbox}
\hspace{0.2cm}


%===================================
\mysection{\textbf{Port Security: Violation Modes}}

\hspace{0.2cm}
\begin{tcolorbox}[width=6.3in]
\scriptsize 
\begin{itemize}
  \item{Protect}  When the number of secure MAC addresses reaches the limit allowed on the port, packets with unknown source addresses are dropped until a sufficient number of secure MAC addresses are removed, or the number of maximum allowable addresses is increased. There is no notification that a security violation has occurred.
  \item{Restrict}  When the number of secure MAC addresses reaches the limit allowed on the port, packets with unknown source addresses are dropped until a sufficient number of secure MAC addresses are removed, or the number of maximum allowable addresses is increased. In this mode, there is a *notification that a security violation has occurred*.
  \item{Shutdown}  In this ({\bf{default}}) mode, a port security violation causes the 
      \subitem{} interface to immediately become error-disabled and turns off the port LED. It increments the violation counter. 
        \subitem{} When a secure port is in the error-disabled state, it can be
	brought out of this state by entering the
	{\verb$shutdown$}  interface configuration mode
	command followed by the {\verb$no shutdown$} command.
\end{itemize}
\end{tcolorbox}
\hspace{0.2cm}
\normalsize  

%===================================
\mysection{\textbf{Port Security: Configuring}}


\hspace{0.2cm}
\begin{tcolorbox}[width=6.3in]
  {\bf{Default Port Security on a Cisco Catalyst Switch}}
\scriptsize 
  \begin{outline}
    \1 Port security and Sticky address learning: Disabled
    \1 Max number of secure MAC addresses: 1
    \1 Violation mode: Shutdown (Port shuts down when max number of secure MAC addresses exceeded).
  \end{outline}
\end{tcolorbox}
\hspace{0.2cm}
\normalsize  


%===================================
\mysection{\textbf{Ports in Error Disabled State}}


\hspace{0.2cm}
\begin{tcolorbox}[width=6.3in]
\scriptsize 
  \begin{outline}
    \1 When a port is error disabled, it is effectively shut down and no traffic is sent or received on that port.
    \1 Port protocol and link status is changed to down; port LED will turn off; port goes into the error disabled stated.
    \1 {\verb$show interfaces$} identifies port status as {\verb$err-disabled$}
    \1 {\verb$show port-security interface$} shows port status as {\verb$secure-shutdown$}
  \end{outline}
The administrator should determine what caused the security violation before
re-enabling the port. If an unauthorized device is connected to a secure port,
the port should not be re-enabled until the security threat is eliminated. To
re-enable the port, use the {\verb$shutdown$} interface configuration mode
  command (Figure 3). Then, use the {\verb$no shutdown$} interface configuration command to make the port operational.
\end{tcolorbox}
\hspace{0.2cm}
\normalsize  


\end{document}
