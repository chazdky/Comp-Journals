\documentclass[../EngineeringJournal_CDavis.tex]{subfiles}

\begin{document}

%%%%%%%%%%%%%%%%%%%%%%%%%%%%%%%%%%%%%%%%%%%%%%%%%%%%%
%%%%%%%%%%%%%%%%%%%%%%%%%%%%%%%%%%%%%%%%%%%%%%%%%%%%%

\chapter*{Useful Bits: \linebreak[1] DHCP \hspace*{\fill}{2020}}
\noindent\textbf{{Packet Tracer labs} \hspace*{\fill}{\textbf{CIT 167}}}\linebreak[1]
{{Spring 2020} \hspace*{\fill}{Chaz Davis}}                             
\addcontentsline{toc}{chapter}{Useful Bits: DHCP}

%===================================

%===================================
\mysection{\textbf{DHCPv4: Basic Configuration}}


\hspace{0.2cm}
\begin{tcolorbox}[width=6.3in]
Steps:
\begin{itemize}
\item{1.} Exclude IPv4 Addresses: e.g. those assigned to devices that require static addresses - routers, servers, printers, other manually configured devices.
  \begin{verbatim}
  \scriptsize 
    \begin{verbatim}
  ip dhcp excluded-address <single-ip>
  ip dhcp excluded-address <low-address> <high-address>
  \end{verbatim}
  \normalsize  

\item{2.} Configuring a DHCPv4 Pool that will be used for assignment
  \scriptsize 
  \begin{verbatim}
  ip dhcp pool <pool-name>        # Enters DHCPv4 config mode
  \end{verbatim}
  \normalsize  

\item{3.} Configuring Specific Tasks
    \subitem{Required:} 
	\scriptsize 
	\begin{verbatim}
	  network <network-num> [ mask | /prefix-length ]
	  default-router <address [ address2 ... address8 ]
	\end{verbatim}
	\normalsize  

    \subitem{Optional:} 
	\scriptsize 
	\begin{verbatim}
	  dns-server <address> [ address2 .. address8 ]
	  domain-name <name>
	  lease ( days [hours] [minutes] | infinite )
	  netbios-name-server <address> [ address2 ... address8 ]
	\end{verbatim}
	\normalsize  
    \subitem{Example:} 
	\scriptsize 
	\begin{verbatim}
	  ip dhcp excluded-address 192.168.10.1 192.168.10.9
	  ip dhcp excluded-address 192.168.10.254
	  ip dhcp pool LAN-POOL-1
	    network 192.168.10.0 255.255.255.0
	    default-router 192.168.10.1
	    dns-server 192.168.11.5
	    domain-name example.com
	    exit
	\end{verbatim}
\end{itemize}
\end{tcolorbox}
\hspace{0.2cm}
\normalsize  


%===================================
\mysection{\textbf{DHCPv4: Troubleshooting}}


\hspace{0.2cm}
\begin{tcolorbox}[width=6.3in]
\scriptsize 
  \begin{itemize}
\item{1.} Resolve IPv4 address conflicts
\item{2.} Verify physical connectivity
\item{3.} Test with a static IPv4 address
\item{4.} Verify switch port configuration
\item{5.}  Resolve IPv4 address conflicts
  \end{itemize}
  \begin{verbatim}
  show ip dhcp conflict
  \end{verbatim}
\end{tcolorbox}
\hspace{0.2cm}
\normalsize  


%===================================
\mysection{\textbf{DHCPv6: Router COnfiguration}}


\hspace{0.2cm}
\begin{tcolorbox}[width=6.3in]
\scriptsize 
  IPv6 routing must b enabled before a router can send RA messages:
  \begin{verbatim}
  ipv6 unicast-routing
  \end{verbatim}
  RA messages are configured on an individual interface of a router. 
  To re-enable an interface for SLAAC that might have been set to 
  another option, the M and O flags need to be reset to their initial 
  values of 0. This is done using the following interface 
  configuration mode commands:
  \begin{verbatim}
  Router(config-if)# no ipv6 nd managed-config-flag 
  Router(config-if)# no ipv6 nd other-config-flag 
  \end{verbatim}
\end{tcolorbox}
\hspace{0.2cm}
\normalsize  


%===================================
\mysection{\textbf{DHCPv6: Stateless Vs Stateful}}


\hspace{0.2cm}
\begin{tcolorbox}[width=6.3in]
Stateless DHCPv6
\scriptsize 
For stateless DHCPv6, the O flag is set to 1 and the M flag is left at 
the default setting of 0. The O flag value of 1 is used to inform the 
client that additional configuration information is available from a 
stateless DHCPv6 server.

To modify the RA message sent on the interface of a router to indicate 
stateless DHCPv6, use the following command:
  \begin{verbatim}
  Router(config-if)# ipv6 nd other-config-flag 
  \end{verbatim}
\end{tcolorbox}
\hspace{0.2cm}
\normalsize  

\hspace{0.2cm}
\begin{tcolorbox}[width=6.3in]
Stateful DHCPv6
\scriptsize 
The M flag indicates whether or not to use stateful DHCPv6. 
The O flag is not involved. The following command is used to change 
the M flag from 0 to 1 to signify stateful DHCPv6:
  \begin{verbatim}
  Router(config-if)# ipv6 nd managed-config-flag 
  \end{verbatim}
\end{tcolorbox}
\hspace{0.2cm}
\normalsize  

%===================================

\end{document}
