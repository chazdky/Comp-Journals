\documentclass[../EngineeringJournal_CDavis.tex]{subfiles}

\begin{document}

%%%%%%%%%%%%%%%%%%%%%%%%%%%%%%%%%%%%%%%%%%%%%%%%%%%%%
%%%%%%%%%%%%%%%%%%%%%%%%%%%%%%%%%%%%%%%%%%%%%%%%%%%%%

\chapter*{Configuring \linebreak[1] VLANs \hspace*{\fill}{2020}}
\noindent\textbf{{Packet Tracer Labs} \hspace*{\fill}{\textbf{CIT 167}}}\linebreak[1]
{{Spring 2020} \hspace*{\fill}{Chaz Davis}}                             
\addcontentsline{toc}{chapter}{Useful Commands: Configuring VLANs}
%===================================

%===================================
\mysection{\textbf{Configuring VLANs}}


\hspace{0.2cm}
\begin{tcolorbox}[width=6.3in]
{\bf{SSH configuration}}
\scriptsize
\begin{verbatim}
- Configurations are stored in the `flash:vlan.dat` file.
\end{verbatim}
\end{tcolorbox}


%===================================
\mysection{\textbf{Display/Verify VLANs}}

\hspace{0.2cm}
\begin{tcolorbox}[width=6.3in]
\scriptsize 
\begin{verbatim}
show vlan
show vlan brief
show vlan summary
show vlan id <vlan-id>
show vlan name <vlan-name>

show interfaces vlan <vlan-id>
show interfaces <interface-id> switchport
\end{verbatim}
\end{tcolorbox}
\hspace{0.2cm}
\normalsize  


\hspace{0.2cm}
\begin{tcolorbox}[width=6.3in]
Create a VLAN
\scriptsize 
\begin{verbatim}
vlan <vlan-id>
name <vlan-name>
exit

vlan 100,102, 105-107	# Creates multiple VLANs
\end{verbatim}
\end{tcolorbox}
\hspace{0.2cm}
\normalsize  

\hspace{0.2cm}
\begin{tcolorbox}[width=6.3in]
Assigning Ports to VLANs
\scriptsize 
\begin{verbatim}
interface <interface-id>
switchport mode access
switchport access vlan <vlan-id>

interface range <interfaces> can be used to configure multipe interfaces.
\end{verbatim}
\end{tcolorbox}
\hspace{0.2cm}
\normalsize  



\hspace{0.2cm}
\begin{tcolorbox}[width=6.3in]
Changing VLAN Port Membership
\scriptsize 
\begin{verbatim}
interface <interface-id>
no switchport access vlan	# Removes vlan; 
				# not necessary if want to replace
  
\end{verbatim}
\end{tcolorbox}
\hspace{0.2cm}
\normalsize  




\hspace{0.2cm}
\begin{tcolorbox}[width=6.3in]
Deleting VLANs
\scriptsize 
\begin{verbatim}
no vlan <vlan-id>	# Delete single VLAN

delete flash:vlan.dat	# Deletes entire vlan.dat file


- Note: before deleting a VLAN, reassign all member ports 
to a different VLAN first. Any ports that are not moved to an 
active VLAN are unable to communicate with other hosts after the 
VLAN is deleted and until they are assigned to an active VLAN.
\end{verbatim}
\normalsize
Restore switch to factory defaults
\scriptsize
\begin{verbatim}
erase startup-config
delete vlan.dat
reload
\end{verbatim}
\end{tcolorbox}
\hspace{0.2cm}
\normalsize  





\hspace{0.2cm}
\begin{tcolorbox}[width=6.3in]
Configuring VLAN Trunks
\scriptsize 
\begin{verbatim}
interface <interface-id>
switchport mode trunk
switchport trunk native vlan <vlan-id>
switchport trunk allowed vlan <vlan-list>
exit
\end{verbatim}
\normalsize
Resetting the Trunk to Default State
\scriptsize
\begin{verbatim}
interface <interface-id>
no switchport trunk allowed vlan
no switchport trunk native vlan
end
\end{verbatim}
\normalsize
Troubleshooting VLANs
\scriptsize
\begin{verbatim}
show vlan
show mac address-table
show interfaces
show interfaces switchport
show interface <interface-id> switchport
\end{verbatim}
\begin{outline}
\1 Each VLAN must correspond to a unique IP subnet; check IP addressing.
\1 Check that ports belongs to expected VLANs and are active.
\1 Check if inactive VLAN is assigned to a port: `show interfaces switchport`
\end{outline}
\end{tcolorbox}
\hspace{0.2cm}
\normalsize  



\hspace{0.2cm}
\begin{tcolorbox}[width=6.3in]
Troubleshooting Trunks
\scriptsize 
\begin{verbatim}
show interfaces trunk
\end{verbatim}
\begin{outline}
\1 Check whether local and peer native VLANs match (VLAN leaking occurs with mismatch).
\1 Check whether trunk has been established. Cisco Catalyst switch ports use DTP by default; statically configure trunk links whenever possible.
\1 Check status of trunk ports for incorrect port modes
\1 Common configuration errors:
\1 Native VLAN mismatches: inter-VLAN routing issues; security risk.
\1 Trunk mode mismatches: trunk link will not work.
\1 Allowed VLANs on trunks: unexpected or no traffic.
\end{outline}
\end{tcolorbox}
\hspace{0.2cm}
\normalsize  

%======================
\end{document}
