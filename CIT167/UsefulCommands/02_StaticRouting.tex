\documentclass[../EngineeringJournal_CDavis.tex]{subfiles}

\begin{document}

%%%%%%%%%%%%%%%%%%%%%%%%%%%%%%%%%%%%%%%%%%%%%%%%%%%%%
%%%%%%%%%%%%%%%%%%%%%%%%%%%%%%%%%%%%%%%%%%%%%%%%%%%%%

\chapter*{Configuring \linebreak[1] Static Routes \hspace*{\fill}{2020}}
\noindent\textbf{{Packet Tracer Labs} \hspace*{\fill}{\textbf{CIT 167}}}\linebreak[1]
{{Spring 2020} \hspace*{\fill}{Chaz Davis}}                             
\addcontentsline{toc}{chapter}{Useful Commands: Static Routing}
%===================================
%===================================

%===================================
\mysection{\textbf{Configuring Static Routes }}

\hspace{0.2cm}
\begin{tcolorbox}[width=6.3in]
  \begin{outline}
    \1 There are two common types of static routes in the routing table:
      \2 Static route to a specific network
      \2 Default static route
  \end{outline}
\end{tcolorbox}
\hspace{0.2cm}

%===================================
\mysection{\textbf{IPv4 Routing}}

% \item{show ip route [ip-address]} to display the current state of the routing table
%   \subitem{command can be used in EXEC or privileged EXEC mode}
% \item{show arp} To display the 
\hspace{0.2cm}
\begin{tcolorbox}[width=6.3in]
\scriptsize
\begin{verbatim}

ip route <network-address> <subnet-mask> { next-hop-ip | exit-intf }

- Next-hop-ip: IP-address of the connecting router to use for forwarding.
- Exit-intf: the outgoing interface to use to forward the packet to the next hop.

\end{verbatim}
\normalsize
\end{tcolorbox}
\hspace{0.2cm}




%===================================
\mysection{\textbf{Configure a IPv4 Default Static Route}}

\hspace{0.2cm}
\begin{tcolorbox}[width=6.3in]
\scriptsize
\begin{verbatim}

ip route 0.0.0.0 0.0.0.0 { exit-intf | next-hop-ip }


\end{verbatim}
\normalsize
The distance parameter is used to create a floating static route by setting an administrative distance that is higher than a dynamically learned route.
\end{tcolorbox}
\hspace{0.2cm}

%===================================
\mysection{\textbf{Configure an IPv4 Floating Static Route}}

\hspace{0.2cm}
\begin{tcolorbox}[width=6.3in]
  - Given an example where R1 is attached to R2 (172.16.2.2) and R3 (10.10.10.2), the following would create a default static route to R2, and a floating static route to R3:
\scriptsize
\begin{verbatim}

ip route 0.0.0.0 0.0.0.0 172.16.2.2     # Default route to R2
ip route 0.0.0.0 0.0.0.0 10.10.10.2 5   # Floating default route to R3

\end{verbatim}
\normalsize
\begin{outline}

\1 Default route to R2 has no administrative distance specified, so would default to 1. This is the preferred route..
\1 Floating default route to R3 has administrative distance 5. Since this value is greater that of the default route, this route "floats" - it is not present in the routing table unless the preferred route fails.

\end{outline}
\end{tcolorbox}
\hspace{0.2cm}

%===================================
\mysection{\textbf{Default Administrative Distances}}

\hspace{0.2cm}
\begin{tcolorbox}[width=6.3in]
\scriptsize
\begin{verbatim}

Connected                   0
Static                      1
EIGRP summary route         5
External BGP                20
Internal EIGRP              90
IGRP                        100
OSPF                        110
IS-IS                       115
RIP                         120
External EIGRP              170
Internal BGP                200

\end{verbatim}
\normalsize
\end{tcolorbox}
\hspace{0.2cm}

%===================================
\mysection{\textbf{Verify a Static Route}}

\hspace{0.2cm}
\begin{tcolorbox}[width=6.3in]
\scriptsize
\begin{verbatim}

show ip route
show ip route static                  # Displays contents of static routes
show ip route static | begin Gateway
show ip route <network>


show running-config | section ip route

# displays the routes that packets will actually take when traveling to their
destination
traceroute <ip-address>
trace <destination>

#displays detailed info about neighboring devices descovered using Cisco Discovery
Protocol
show cdp neighbors
show cdp neighbors detail

\end{verbatim}
\normalsize
\end{tcolorbox}
\hspace{0.2cm}

%===================================
\mysection{\textbf{Command Sequence}}

\hspace{0.2cm}
\begin{tcolorbox}[width=6.3in]
\scriptsize
\begin{verbatim}

enable
configure terminal

ip route 192.168.3.0 255.255.255.0 192.168.2.2
exit

show ip route
ping 192.168.3.1
trace 192.168.3.1

exit

\end{verbatim}
\normalsize
\end{tcolorbox}
\hspace{0.2cm}

%===================================
\mysection{\textbf{IPv6 Routing}}

\hspace{0.2cm}
\begin{tcolorbox}[width=6.3in]
\scriptsize
\begin{verbatim}

ipv6 route <ipv6-prefix>/<prefixlength> {ipv6-address | exit-intf}


\end{verbatim}
\normalsize
\end{tcolorbox}
\hspace{0.2cm}

%===================================
\mysection{\textbf{Displaying and Testing IPv6 Routes}}

\hspace{0.2cm}
\begin{tcolorbox}[width=6.3in]
\scriptsize
\begin{verbatim}

show ipv6 route
show ipv6 route static
show ipv6 route <network>
show running-config | section ipv6 route

ping ipv6 <ipv6-address>
traceroute

\end{verbatim}
\normalsize
\end{tcolorbox}
\hspace{0.2cm}

%===================================
\mysection{\textbf{Configure a Directly Connected Static IPv6 Route}}

\hspace{0.2cm}
\begin{tcolorbox}[width=6.3in]
\scriptsize
\begin{verbatim}

ipv6 route <ipv6-address>/<network-prefix> <interface>
ipv6 route 2001:db8:acad:2::/64 s0/0/0

\end{verbatim}
\normalsize
\end{tcolorbox}
\hspace{0.2cm}

\clearpage

%===================================
\mysection{\textbf{Configure a Fully Specified Static IPv6 Route}}

\hspace{0.2cm}
\begin{tcolorbox}[width=6.3in]
\scriptsize
\begin{verbatim}

ipv6 route <ipv6-address>/<network-prefix> <interface> <next-hop ipv6-address>
ipv6 route 2001:db8:acad:2::/64 s0/0/0 fe80::2

\end{verbatim}
\normalsize
\end{tcolorbox}
\hspace{0.2cm}

%===================================
\mysection{\textbf{Configure a Default IPv6 Static Route}}

\hspace{0.2cm}
\begin{tcolorbox}[width=6.3in]
\scriptsize
\begin{verbatim}

ipv6 route ::/0 {ipv6-address | exit-intf}
ipv6 route ::/ 2001:db8:acad:4::2

\end{verbatim}
\normalsize
\end{tcolorbox}
\hspace{0.2cm}

%===================================
\mysection{\textbf{Configure an IPv6 Floating Static Route}}


\hspace{0.2cm}
\begin{tcolorbox}[width=6.3in]
\scriptsize
\begin{verbatim}

ipv6 route ::/0 <ipv6-address> <administrative-distance>
ipv6 route ::/0 2001:db8:aad:4::2
ipv6 route ::/0 2001:db8:aad:6::2 5


\end{verbatim}
\normalsize
\end{tcolorbox}
\hspace{0.2cm}

%===================================
\mysection{\textbf{Common IOS Troubleshooting Commands}}


\hspace{0.2cm}
\begin{tcolorbox}[width=6.3in]
\scriptsize
\begin{verbatim}

ping
ping <ip-address> source <source_ip>
traceroute <ip-address>
show ip route | begin Gateway
show ip interface brief
show cdp neighbors detail


\end{verbatim}
\normalsize
\end{tcolorbox}
\hspace{0.2cm}

\newpage

%===================================

\end{document}
