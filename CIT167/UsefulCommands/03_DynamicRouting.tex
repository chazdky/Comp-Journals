\documentclass[../EngineeringJournal_CDavis.tex]{subfiles}

\begin{document}

%%%%%%%%%%%%%%%%%%%%%%%%%%%%%%%%%%%%%%%%%%%%%%%%%%%%%
%%%%%%%%%%%%%%%%%%%%%%%%%%%%%%%%%%%%%%%%%%%%%%%%%%%%%

\chapter*{Configuring \linebreak[1] Dynamic Routing \hspace*{\fill}{2020}}
\noindent\textbf{{Packet Tracer Labs} \hspace*{\fill}{\textbf{CIT 167}}}\linebreak[1]
{{Spring 2020} \hspace*{\fill}{Chaz Davis}}                             
\addcontentsline{toc}{chapter}{Useful Commands: Dynamic Routing}
%===================================

%=================================
\mysection{Access RIP configuration mode}

\hspace{0.2cm}
\begin{tcolorbox}[width=6.3in]
{\bf{router rip}}
\scriptsize
\begin{outline}
  \1 Prompt will show up as \verb$Router(config-router)#$
  \1 When enabling RIP, the default version is RIPv1.
  \1 To disable and eliminate RIP, use the `no router rip` global configuration command. This command stops the RIP process and erases all existing RIP configurations.
  \1 Upon enabling, need to advertise networks with `network <network-address>`
\end{outline}
\end{tcolorbox}
\hspace{0.2cm}
\normalsize  

%=================================

\hspace{0.2cm}
\begin{tcolorbox}[width=6.3in]
{\bf{Enable and Verify RIPv2}}
\scriptsize

By {\emph{default, when a RIP process is configured on a Cisco router, it is running
RIPv1}}, as shown in Figure 1. However, even though the router only sends RIPv1
messages, it can interpret both RIPv1 and RIPv2 messages. A {\emph{RIPv1 router ignores
the RIPv2 fields in the route entry.}}

Use the `version 2` router configuration mode command to enable RIPv2, as shown in
Figure 2. Notice how the `show ip protocols` command verifies that R2 is now configured
to send and receive version 2 messages only. The RIP process now includes the subnet
mask in all updates, making {\emph{RIPv2 a classless routing protocol.}}

Note: Configuring `version 1` enables RIPv1 only, while configuring `no version` returns the router to the default setting of sending version 1 updates but listening for version 1 and version 2 updates.

\end{tcolorbox}
\hspace{0.2cm}
\normalsize  

%=================================

\hspace{0.2cm}
\begin{tcolorbox}[width=6.3in]
  {\bf{Propogate a Default Route}}
\scriptsize
\begin{verbatim}

 The `default-information originate` router configuration command. 
 This instructs R1 to originate default information, 
 by propagating the static default route in RIP updates.

 \end{verbatim}
\end{tcolorbox}
\hspace{0.2cm}
\normalsize  

%=================================

\hspace{0.2cm}
\begin{tcolorbox}[width=6.3in]
  {\bf{RIP Routing Configuration Mode Commands}}
\scriptsize
\begin{verbatim}

```
network <network-address>       # Enables RIP on all interfaces on that network.
version 2                       # Enables RIPv2
version 1                       # Enables RIPv1 only
no version                      # Returns to default setting: send v1, listen v1+v2
no auto-summary                 # Disables default automatic summarization (RIPv2)

passive-interface <interface>   # Prevent transmission of routing updates out interface
				# However, still allows network to be advertised
passive-interface g0/0
passive-interface default       # Makes all interfaces passive
no passive-interface <intf>     # Re-enables passive transmission of routing updates
```
\end{verbatim}
\end{tcolorbox}
\hspace{0.2cm}
\normalsize  

%=================================

\hspace{0.2cm}
\begin{tcolorbox}[width=6.3in]
{\bf{Verify RIP Routing}}
\scriptsize
\begin{verbatim}
```
show ip protocols
show ip route
```
\end{verbatim}
\end{tcolorbox}
\hspace{0.2cm}
\normalsize  

%=================================

\hspace{0.2cm}
\begin{tcolorbox}[width=6.3in]
{\bf{ Propogate a Default Route }}
\scriptsize
\begin{verbatim}
To propagate a default route in RIP, 
the edge router must be configured with a default static route 
using the `ip route 0.0.0.0 0.0.0.0` command:

```
ip route 0.0.0.0 0.0.0.0 <exit-interface> <next-hop IP>
```

Next configure router to propogate the static default route in RIP updates:

```
router rip
default-information originate
```
\end{verbatim}
\end{tcolorbox}
\hspace{0.2cm}
\normalsize  
%===================================

\end{document}
