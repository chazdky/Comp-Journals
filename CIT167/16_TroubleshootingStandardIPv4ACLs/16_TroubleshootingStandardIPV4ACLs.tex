\documentclass{report}

\usepackage{textcomp}
\usepackage[demo]{graphicx}
\usepackage{graphicx}
\usepackage{fancyhdr}
\usepackage{subcaption}
\usepackage{multicol}
\usepackage{outlines}
%===================================
\newcommand{\classinfo}{{\bf Troubleshooting Standard \\ IPV4 ACLs}\\{\it CIT 167 \\ Chaz Davis}}
\newcommand{\semester}{BCTC \\ Spring 2020}
%===================================
\newcommand{\mysection}[1]{\section*{#1}}
\newcommand{\mysubsection}[2]{\textbf{\romannumeral #1) #2}}
%===================================
\setlength{\headheight}{15.2pt}
\pagestyle{fancy}
\fancyhf{}
\lhead{ \fancyplain{}{Chaz Davis} }
\rhead{ \fancyplain{}{\today} }
\cfoot{ \fancyplain{}{\thepage} }
\renewcommand{\headrulewidth}{0.5pt}
\renewcommand{\footrulewidth}{0pt}

%===================================
\title{\classinfo}
\author{\semester}
\date{\today}

%===================================

\begin{document}

\maketitle

%===================================
\mysection{\textbf{Part 1: Troubleshoot ACL Issue 1}}

\mysubsection{1}{Determine ACL problem}\\
I realized that the Servers were misconfigured and went in to 
redo all of their addresses.
See Fig.~\ref{P1Server16}\subref{P1Server16Serv1} through
Fig.~\ref{P1Server16}\subref{P1Server16Serv3} on
Pg.~\pageref{P1Server16}.


\begin{figure}[!hbt]\centering
\subfloat[IP Config for Server1]{\label{P1Server16Serv1}\includegraphics[width=.45\linewidth]{demo}}\hfill
\subfloat[IP Config for server2]{\label{P1Server16Serv2}\includegraphics[width=.45\linewidth]{demo}}\par 
\subfloat[IP Config for server3]{\label{P1Server16Serv3}\includegraphics[width=.45\linewidth]{demo}}
\caption{IP Configurations for the servers}\label{P1Server16}
\end{figure}


Now, the first step is to check if LAN1 is denied access to LAN2,
so from L1 I will ping Server2. The successful output of that is in
Fig.~\ref{P1Net16}\subref{P1Net16Ping1} on Pg.~\pageref{P1Net16}. 


We are also told that Lan3 should have access to Lan2, so 
from L3 I will ping Server2. The ping was blocked by the router
,see Fig.~\ref{P1Net16}\subref{P1Net16Ping2} on Pg.~\pageref{P1Net16}.
Which means that Lan1 has access to Lan2 and Lan3 is blocked from Lan3. 


\begin{figure}[!hbt]\centering
\subfloat[L1 Pinging Server2]{\label{P1Net16Ping1}\includegraphics[width=.45\linewidth]{demo}}\hfill
\subfloat[L3 Pinging Server2]{\label{P1Net16Ping2}\includegraphics[width=.45\linewidth]{demo}}\par 
\caption{Assessing the Network configs for DENY-LAN1 on R1}\label{P1Net16}
\end{figure}


\noindent\mysubsection{2}{Implement a Solution}\\
I logged into R1, I looked up the access tables, I went to
{\scriptsize{\verb$int g0/1$}\normalsize} and ran
{\scriptsize{\verb$no ip access-group DENY-LAN1 out$}\normalsize}. I then
reconfigured DENY-LAN1 to say 
{\scriptsize{\verb$20 permit any any$}\normalsize} that way it would 
allow all other addresses.  I then went to 
{\scriptsize{\verb$int g0/1$}\normalsize} and typed
in {\scriptsize{\verb$ip access-group DENY-LAN1 in$}\normalsize}. See
Fig.~\ref{P1Config16} on Pg.~\pageref{P1Config16} for outputs.


\begin{figure}[!hbt]\centering
\subfloat[Configuring the ACL DENY-LAN1]{\label{P1Config16Deny1}\includegraphics[width=.45\linewidth]{demo}}\hfill
\subfloat[Output of the Access-lists]{\label{P1Config16Deny2}\includegraphics[width=.45\linewidth]{demo}}\par 
\subfloat[G0/1 Configured with DENY-LAN1]{\label{P1Config16Deny3}\includegraphics[width=.45\linewidth]{demo}}
\caption{Configuring the ACL for R1 DENY-LAN1}\label{P1Config16}
\end{figure}



\noindent\mysubsection{3}{Verify that the problem is resolved and document the
solution}\\
I Verified the network connections by running a ping from L1 to
Server2(Fig.~\ref{P1Verify16}\subref{P1Verify16L1}), and
again from L3 to Server2(Fig.~\ref{P1Verify16}\subref{P1Verify16L3}).


\begin{figure}[!hbt]\centering
\subfloat[L1 Ping to Server2]{\label{P1Verify16L1}\includegraphics[width=.45\linewidth]{demo}}\hfill
\subfloat[L3 Ping to Server2]{\label{P1Verify16L3}\includegraphics[width=.45\linewidth]{demo}}\par 
\caption{Verifying DENY-LAN1 Connections}\label{P1Verify16}
\end{figure}


\clearpage

%===================================
\mysection{\textbf{Part 2: Troubleshoot ACL Issue 2}}


\mysubsection{1}{Determine the ACL problem}\\
The second thing we are told is that L2 should have access to LAN3. But, that 
Lan2 shouldn't have access to Lan3.  So, I will ping Server3 from L2. 
Next, I will ping Server3 from Server2. Both were unsuccessful. See Fig.~\ref{P2Net16}\subref{P2Net16Ping1} 
and Fig.~\ref{P2Net16}\subref{P2Net16Ping2}
on Pg.~\pageref{P2Net16} for those outputs.


Lastly, I ran the commands to show the output of access lists, 
to see how DENY-L2 was configured. See Fig.~\ref{P1Net16}\subref{P2Net16R1}.


\begin{figure}[!hbt]\centering
\subfloat[Pinging Server3 from L2]{\label{P2Net16Ping1}\includegraphics[width=.45\linewidth]{demo}}\hfill
\subfloat[Pinging Server3 from Server2]{\label{P2Net16Ping2}\includegraphics[width=.45\linewidth]{demo}}\par 
\subfloat[DENY-L2 Access-list]{\label{P2Net16R1}\includegraphics[width=.45\linewidth]{demo}}
\caption{Determining the problems in part 2}\label{P2Net16}
\end{figure}

\noindent\mysubsection{2}{Implement a Solution}\\
I logged into R1, I looked up the access tables, I went to
{\scriptsize{\verb$int g0/2$}\normalsize} and ran
{\scriptsize{\verb$no ip access-group DENY-L2 out$}\normalsize}. I then
reconfigured DENY-L2 to say 
{\scriptsize{\verb$10 deny host 192.168.0.2$}\normalsize} 
{\scriptsize{\verb$20 permit any$}\normalsize} that way it would 
allow all other addresses.  I then went to 
{\scriptsize{\verb$int g0/1$}\normalsize} and typed
in {\scriptsize{\verb$ip access-group DENY-L2 in$}\normalsize}. See
Fig.~\ref{P2Config16} on Pg.~\pageref{P2Config16} for outputs.


\begin{figure}[!hbt]\centering
\subfloat[Configuring the ACL DENY-L2]{\label{P2Config16Deny1}\includegraphics[width=.45\linewidth]{demo}}\hfill
\subfloat[Output of the Access-lists]{\label{P2Config16Deny2}\includegraphics[width=.45\linewidth]{demo}}\par 
\subfloat[G0/2 Configured with DENY-L2]{\label{P2Config16Deny3}\includegraphics[width=.45\linewidth]{demo}}
\caption{Configuring the ACL for R1 DENY-L2}\label{P2Config16}
\end{figure}


\noindent\mysubsection{3}{Verify that the problem is resolved and document the
solution}\\
I Verified the network connections by running a ping from L2 to
Server3(Fig.~\ref{P2Verify16}\subref{P2Verify16L2}), and
again from Server2 to Server3(Fig.~\ref{P2Verify16}\subref{P2Verify16Serv2}).


\begin{figure}[!hbt]\centering
\subfloat[L2 Ping to Server3]{\label{P2Verify16L2}\includegraphics[width=.45\linewidth]{demo}}\hfill
\subfloat[Server2 Ping to Server3]{\label{P2Verify16Serv2}\includegraphics[width=.45\linewidth]{demo}}\par 
\caption{Verifying DENY-L2 Connections}\label{P2Verify16}
\end{figure}


\clearpage


%===================================
\mysection{\textbf{Part 3: Troubleshoot ACL Issue 3}}


\mysubsection{1}{Determine the ACL problem}\\
Finally, we will test the connection from L3 and attempt to ping both L1, which
it should have access to, and then to L2, which it should also have access to.
It appears that both of theses attempts were also blocked at the router. See
You can see in Fig.~\ref{P1Net16}~\subref{P1Net16Ping3} 
on Pg.~\pageref{P1Net16}.



\noindent\mysubsection{2}{Implement a Solution}\\
a

\noindent\mysubsection{3}{Verify that the problem is resolved and document the
solution}\\
a


%===================================

\end{document}
