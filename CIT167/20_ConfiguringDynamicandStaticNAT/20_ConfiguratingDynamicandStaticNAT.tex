\documentclass{report}

\usepackage{textcomp}
\usepackage[demo]{graphicx}
\usepackage{graphicx}
\usepackage{fancyhdr}
\usepackage{subcaption}
\usepackage{multicol}
\usepackage{outlines}
%===================================
\newcommand{\classinfo}{{\bf Configuring Dynamic \\ and Static NAT}\\{\it CIT 167}\\{Chaz Davis}}
\newcommand{\semester}{BCTC \\ Spring 2020}
%===================================
\newcommand{\mysection}[1]{\section*{#1}}
\newcommand{\mysubsection}[2]{\textbf{\romannumeral #1) #2}}
%===================================
\setlength{\headheight}{15.2pt}
\pagestyle{fancy}
\fancyhf{}
\lhead{ \fancyplain{}{Chaz Davis} }
\rhead{ \fancyplain{}{\today} }
\cfoot{ \fancyplain{}{\thepage} }
\renewcommand{\headrulewidth}{0.5pt}
\renewcommand{\footrulewidth}{0pt}

%===================================
\title{\classinfo}
\author{\semester}
\date{\today}

%===================================

\begin{document}

\maketitle

%===================================
\mysection{\textbf{Part 1: Build the Network and Verify Connectivity}}

\mysubsection{1}{Cable the network as shown in the topology}\\
I set up the network according to the topology, You can see in
Fig.~\ref{P1Top20}\subref{P1Top20a} on Pg.~\pageref{P1Top20}. 


\noindent\mysubsection{2}{Configure PC Hosts}\\
Next, I configured the PCs according to the chart. See
Fig.~\ref{P1Top20}\subref{P1Top20b} and Fig.~\ref{P1Top20}\subref{P1Top20c} on 
Pg.~\pageref{P1Top20}.


\begin{figure}[!hbt]\centering
\subfloat[The topology of the network]{\label{P1Top20a}\includegraphics[width=.45\linewidth]{demo}}\par
\subfloat[Configuring PC-A]{\label{P1Top20b}\includegraphics[width=.45\linewidth]{demo}}\hfill
\subfloat[Configuring PC-B]{\label{P1Top20c}\includegraphics[width=.45\linewidth]{demo}}\par
\caption{Toppology of network and PC configurations}
\label{P1Top20}
\end{figure}

\noindent\mysubsection{3}{Initialize and reload the routers and switches}\\
I initialized the routers and Switches.

\noindent\mysubsection{4}{Configure basic settings for each router}\\
I logged into the routers and configured each of them.

\noindent\mysubsection{5}{Create a simulated web server on ISP}\\
a

\noindent\mysubsection{6}{Configure Static routing}\\
a

\noindent\mysubsection{7}{Save the running Config to the startup config}\\
a

\noindent\mysubsection{8}{Verify the network connectivity}\\
a




%===================================
\mysection{\textbf{Part 2: Configure and Verify Static NAT}}

\mysubsection{1}{Configure static Mapping}\\
a

\noindent\mysubsection{2}{Specify the Interfaces}\\
a

\noindent\mysubsection{3}{Test the Configuration}\\
a

{\bf{a.}}\\
{\bf{What is the translation of the Inside local host?}}\\
{\bf{192.168.1.20=}} 209.165.200.225


{\bf{The Inside global address is assigned by?}} By the ISP and the NAT pool


{\bf{The inside local address is assigned by?}} by the Administrator

{\bf{b.}}
{\bf{What port number was used in this ICMP exchange?}} we can see that port 5
was used the first time, followed by port 6, then port 7, and finally port 8.

{\bf{c.}}
{\bf{What was the protocol used in this translation?}} TCP


{\bf{What are the port numbers used?}} Ports 1025 and 1026 were used from the
inside Because I ran once with the SSH/TELNET client, and then I ran in from
the commandline as well using the telnet command. And port 23 was used to reach the telnet on the loopback interface.

{\bf{Inside global/local:}}\\
Inside global is 209.165.200.225:1025 and Inside local is 192.168.1.20:1025


{\bf{Outside global/local:}}\\
Outside Global is 192.31.7.1:23 and outside local is also 192.31.7.1:23

{\bf{d.}}

{\bf{e.}}

{\bf{f.}}



%===================================
\mysection{\textbf{Part 3: Configure and Verify Dynamic NAT }}

\mysubsection{1}{Clear NATs}\\
a
\noindent\mysubsection{2}{Define an ACL that matches the LAN private IP address
range}\\
a
\noindent\mysubsection{3}{Verify that the NAT interface configurations are
still valid}\\
a
\noindent\mysubsection{4}{Define the pool of usable public IP addresses}\\
a
\noindent\mysubsection{5}{Define the NAT from the inside source list to the
outside pool}\\
a
\noindent\mysubsection{6}{Test the configuration}\\
a
{\bf{a.}}\\
{\bf{What is the translation of the Inside local host address for PC-B?}}
{\bf{192.168.1.21=}} 209.165.200.242


{\bf{A dynamic NAT entry was added to the table with ICMP as the protocol when PC-B sent an ICMP
message to 192.31.7.1 on ISP}}


{\bf{What port number was used in this ICMP exchange?}} ports 5,6,7, and 8


{\bf{b.}}\\
{\bf{From PC-B, open a browser and enter the IP address of the ISP-simulated web server (Lo0 interface).
When prompted, log in as webuser with a password of webpass.}}



{\bf{c.}}\\
{\bf{What protocol was used in this translation? }} TCP


{\bf{What port numbers were used?}}\\
{\bf{Inside: }} ports 1025 -- 1028


{\bf{Outside: }} port 80


{\bf{What well-known port number and service was used?}} port 80, which is HTTP
port.


{\bf{d.}}

\noindent\mysubsection{7}{Remove the static NAT entry}\\
a


%===================================
\mysection{\textbf{Reflection}}

\mysubsection{1}{Why would NAT be used in a network?}\\
Thee won't be enough ipv4 addresses, so, NAT provides a way to ensure that we
have the available resources and addresses to match. Secondly, NAT ensures a
higher level of security, we can hide resources that we don't want to be
reachable by the outside network.

\noindent\mysubsection{2}{What are the limitations of NAT?}\\
NAT needs ip information or port information in the ip header or TCP header for translations.
Also, End-to-End addressing is lost. So, certain peer-to-peer applications may
not work. Switching delays may occur. It complicates
tunneling protocols. Impropoer use of a network layer device, router, tampering
with port numbers.
  
%===================================

\end{document}
