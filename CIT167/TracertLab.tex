\document{article}

\begin{document}

\title{1.1.1.8 - Trace Route: discover the network}

\author{Chaz Davis}

\maketitle

\section{Getting IP Tables}
\subsection{Sales Machine}
I clicked on the sales machine, clicked desktop, and when i opened the command prompt i ran ip config.
% TODO:  <23-01-20, yourname> % insert screen shot of ipconfig here

the sales ip address was 172.16.0.0
\subsection{b2server}
i ran the nslookup on b2server
its address was:
Non-authoritative answer:
Name:   b2server.pt.pka
Address:   128.107.64.254
\subsubsection{tracert}
i then ran a tracert between b2server and sales pc
Tracing route to 128.107.64.254 over a maximum of 30 hops: 

  1   10 ms     0 ms      0 ms      172.16.0.1
  2   1 ms      1 ms      1 ms      64.100.150.2
  3   0 ms      1 ms      1 ms      64.104.222.2
  4   12 ms     10 ms     10 ms     64.104.222.6
  5   *         13 ms     11 ms     128.107.64.254

\subsubsection{telnet}
i then telnetted to the first ip address in the tracert: 172.16.0.1
i entered the passwd cisco to gain access. i the logged in as priveledged exec with passwd class

\subsection{R4}
\subsubsection{tracert from R4}
when you run the tracert command from the router you have to spell the entire command out traceroute
R4>traceroute 128.107.64.254
Type escape sequence to abort.
R4#traceroute 128.107.64.254
Type escape sequence to abort.
Tracing the route to 128.107.64.254

  1   64.100.150.2    1 msec    0 msec    0 msec    
  2   64.104.222.2    0 msec    1 msec    1 msec    
  3   64.104.222.6    0 msec    0 msec    0 msec    
  4   128.107.64.254  1 msec    2 msec    4 msec    

\subsubsection{show ip interface brief}
i ran show ip interface brief on R4 router:
Interface              IP-Address      OK? Method Status                Protocol 
GigabitEthernet0/0     172.16.0.1      YES manual up                    up 
GigabitEthernet0/1     unassigned      YES unset  administratively down down 
Serial0/0/0            64.100.150.1    YES manual up                    up 
Serial0/0/1            unassigned      YES unset  up                    up 
Serial0/0/1.1          64.100.200.1    YES manual up                    up 
Vlan1                  unassigned      YES unset  administratively down down

\subsubsection{show run}

I then ran the show run command
Building configuration...

Current configuration : 1918 bytes
!
version 15.1
no service timestamps log datetime msec
no service timestamps debug datetime msec
service password-encryption
!
hostname R4
!
!
!
enable password 7 0822404F1A0A
!
!
ip dhcp excluded-address 172.16.0.1
ip dhcp excluded-address 172.16.0.2
ip dhcp excluded-address 172.16.0.3
ip dhcp excluded-address 172.16.0.4
!
!
!
!
ip cef
no ipv6 cef
!
!
!
username Tier3a password 7 0822455D0A16
username administrator password 7 0822455D0A16
!
!
license udi pid CISCO1941/K9 sn FTX1524OSTD
!
!
!
!
!
!
!
!
!
ip domain-name Branch
ip host branchserver 172.16.0.3 
!
!
spanning-tree mode pvst
!
!
!
!
!
!
interface GigabitEthernet0/0
 ip address 172.16.0.1 255.255.255.0
 ip nat inside
 duplex auto
 speed auto
!
interface GigabitEthernet0/1
 no ip address
 duplex auto
 speed auto
 shutdown
!
interface Serial0/0/0
 ip address 64.100.150.1 255.255.255.252
 encapsulation ppp
 ppp authentication chap
 ip nat outside
!
interface Serial0/0/1
 no ip address
 encapsulation frame-relay
 ip nat outside
 clock rate 2000000
!
interface Serial0/0/1.1 point-to-point
 ip address 64.100.200.1 255.255.255.252
 frame-relay interface-dlci 201
 ip nat outside
 clock rate 2000000
!
interface Vlan1
 no ip address
 shutdown
!
ip nat inside source list 1 interface Serial0/0/1.1 overload
ip nat inside source static tcp 172.16.0.3 80 64.100.150.1 80 
ip nat inside source static tcp 172.16.0.3 80 64.100.200.1 80 
ip nat inside source static tcp 172.16.0.3 25 64.100.200.1 25 
ip nat inside source static tcp 172.16.0.3 443 64.100.200.1 443 
ip classless
ip route 10.0.0.0 255.0.0.0 64.100.200.2 
ip route 0.0.0.0 0.0.0.0 64.100.150.2 
!
ip flow-export version 9
!
!
access-list 1 permit 172.16.0.0 0.0.0.255
!
no cdp run
!
banner motd ^CWarning - Unauthorized Access Prohibited^C
!
!
!
!
!
line con 0
!
line aux 0
!
line vty 0 4
 exec-timeout 0 0
 password 7 0822455D0A16
 login
line vty 5 15
 exec-timeout 0 0
 no login
!
!
!
end

\subsection{serial0/0/0}
i then used tel net to get to serial0/0/0 using the ip adress  64.100.150.1
i looged in to exec and then priveledged exec

\subsubsection{show ip route}
i issued the show ip route command:

Codes: L - local, C - connected, S - static, R - RIP, M - mobile, B - BGP
       D - EIGRP, EX - EIGRP external, O - OSPF, IA - OSPF inter area
       N1 - OSPF NSSA external type 1, N2 - OSPF NSSA external type 2
       E1 - OSPF external type 1, E2 - OSPF external type 2, E - EGP
       i - IS-IS, L1 - IS-IS level-1, L2 - IS-IS level-2, ia - IS-IS inter area
       * - candidate default, U - per-user static route, o - ODR
       P - periodic downloaded static route

Gateway of last resort is 64.100.150.2 to network 0.0.0.0

S    10.0.0.0/8 [1/0] via 64.100.200.2
     64.0.0.0/8 is variably subnetted, 5 subnets, 2 masks
C       64.100.150.0/30 is directly connected, Serial0/0/0
L       64.100.150.1/32 is directly connected, Serial0/0/0
C       64.100.150.2/32 is directly connected, Serial0/0/0
C       64.100.200.0/30 is directly connected, Serial0/0/1.1
L       64.100.200.1/32 is directly connected, Serial0/0/1.1
     172.16.0.0/16 is variably subnetted, 2 subnets, 2 masks
C       172.16.0.0/24 is directly connected, GigabitEthernet0/0
L       172.16.0.1/32 is directly connected, GigabitEthernet0/0
S*   0.0.0.0/0 [1/0] via 64.100.150.2



\end{document}
