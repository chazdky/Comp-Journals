\documentclass[../EngineeringJournal_CDavis.tex]{subfiles}

\begin{document}

%%%%%%%%%%%%%%%%%%%%%%%%%%%%%%%%%%%%%%%%%%%%%%%%%%%%%
%%%%%%%%%%%%%%%%%%%%%%%%%%%%%%%%%%%%%%%%%%%%%%%%%%%%%

\chapter[Documenting The Network]{Documenting \linebreak[1] The Network \hspace*{\fill}{Jan 26, 2020}}
\noindent\textbf{{Packet Tracer Lab 2} \hspace*{\fill}{\textbf{CIT 167}}}\linebreak[1]
{{Spring 2020} \hspace*{\fill}{Chaz Davis}}                             
%===================================
%===================================

\hspace{0.2cm}
\begin{tcolorbox}[width=6.3in]
\scriptsize 
- Important Commands for the Lab
  \begin{itemize}
	\item{show interface} displays the status of the router's interfaces
	  \subitem{interface status} up/down
	  \subitem{Protocol status on the device}
	  \subitem{Utilization}
	  \subitem{Errors}
	  \subitem{MTU}
	  \subitem{MTU}
	\item{show ip interface (brief)} tons of useful info:
	  \subitem{IP protocol status}
	  \subitem{all its services}
	  \subitem{all of its interfaces} and their protocols and status'
	  \subitem{Ip address}
	  \subitem{Layer 2 status}
	  \subitem{layer 3 status}
	  \subitem{} it is the config thats in the routers memory.
	  \subitem{sh run} this is not updated upon changes. but upon running the command copy running-configuration startup-configuration
	\item{show ip route} 
	  \subitem{}lists all networks that the router can reach 
	  \subitem{}their metric - the routers preference for them and how to get there
	\item{show version} gives the routers config register (firmware settings)
	  \subitem{the last time it was booted}
	  \subitem{the version of the IOS}
	  \subitem{the model of router}
	  \subitem{the amount of ram and Flash}
  \end{itemize}
\normalsize  
\end{tcolorbox}
\hspace{0.2cm}

\newpage

%===================================

\mysection{\textbf{Part 1: filling in the table}}

\mysubsection{1}{R2}\\
I logged into R2 and ran the following commands 
  i ran show ip config

\begin{mdframed}
\scriptsize
\begin{verbatim}
Interface              IP-Address      OK? Method Status Protocol 
GigabitEthernet0/0     10.255.255.245  YES manual up 	     up 
GigabitEthernet0/1     10.255.255.249  YES manual up        up 
GigabitEthernet0/2     10.10.10.1      YES manual up        up 
Serial0/0/0            64.100.100.1    YES manual up        up 
Serial0/0/1            unassigned      YES unset  up        up 
Serial0/0/1.1          64.100.200.2    YES manual up        up 
Vlan1                  unassigned      YES unset
administratively down down
\end{verbatim}
\normalsize
\end{mdframed}

I then ran  show interfaces
\begin{mdframed}
\scriptsize
\begin{verbatim}
GigabitEthernet0/0 is up, line protocol is up (connected)
Hardware is CN Gigabit Ethernet, address is 
0001.969a.1d01 (bia 0001.969a.1d01)
Internet address is 10.255.255.245/30
MTU 1500 bytes, BW 1000000 Kbit, DLY 10 usec,
reliability 255/255, txload 1/255, rxload 1/255
Encapsulation ARPA, loopback not set
Keepalive set (10 sec)
Full-duplex, 100Mb/s, media type is RJ45
output flow-control is unsupported, input flow-control is unsupported
ARP type: ARPA, ARP Timeout 04:00:00, 
Last input 00:00:08, output 00:00:05, output hang never
Last clearing of "show interface" counters never
Input queue: 0/75/0 (size/max/drops); Total output drops: 0
Queueing strategy: fifo
Output queue :0/40 (size/max)
5 minute input rate 63 bits/sec, 0 packets/sec
5 minute output rate 65 bits/sec, 0 packets/sec
241 packets input, 16664 bytes, 0 no buffer
Received 0 broadcasts, 0 runts, 0 giants, 0 throttles
0 input errors, 0 CRC, 0 frame, 0 overrun, 0 ignored, 0 abort
0 watchdog, 1017 multicast, 0 pause input
0 input packets with dribble condition detected

\end{verbatim}
\normalsize
\end{mdframed}

I then ran show running-config
\begin{mdframed}
\scriptsize
\begin{verbatim}
Building configuration...

Current configuration : 2258 bytes
version 15.1
no service timestamps log datetime msec
no service timestamps debug datetime msec
no service password-encryption
hostname R2
enable secret 5 $1$mERr$9cTjUIEqNGurQiFU.ZeCi1
ip cef
no ipv6 cef
username Tier3a password 0 cisco
username administrator password 0 cisco
license udi pid CISCO2911/K9 sn FTX15249169
ip ssh version 2
ip domain-name Central
ip host R1 10.255.255.254 10.255.255.246 10.2.0.1 
ip host R3 10.255.255.253 10.255.255.250 10.3.0.1 
spanning-tree mode pvst
interface GigabitEthernet0/0
 ip address 10.255.255.245 255.255.255.252
 ip ospf priority 255
 duplex auto
 speed auto
interface GigabitEthernet0/1
 ip address 10.255.255.249 255.255.255.252
 ip ospf priority 255
 duplex auto
 speed auto
interface GigabitEthernet0/2
 ip address 10.10.10.1 255.255.255.0
 duplex auto
 speed auto
interface Serial0/0/0
 ip address 64.100.100.1 255.255.255.252
 encapsulation ppp
 ppp authentication pap
 ppp pap sent-username R2 password 0 cisco
interface Serial0/0/1
 no ip address
 encapsulation frame-relay
 clock rate 2000000
interface Serial0/0/1.1 point-to-point
 ip address 64.100.200.2 255.255.255.252
 frame-relay interface-dlci 202
 clock rate 2000000
interface Vlan1
 no ip address
 shutdown
router ospf 1
 log-adjacency-changes
 passive-interface Serial0/0/0
 passive-interface Serial0/0/1
 network 10.255.255.244 0.0.0.3 area 0
 network 10.255.255.248 0.0.0.3 area 0
 network 10.10.10.0 0.0.0.255 area 0
 network 64.100.100.0 0.0.0.3 area 0
 default-information originate
ip classless
ip route 0.0.0.0 0.0.0.0 64.100.100.2 
ip route 0.0.0.0 0.0.0.0 64.100.200.1 200
ip flow-export version 9
ip access-list extended PERMIT_LOCAL
 permit udp any any eq domain
 permit tcp any any eq domain
 permit ip 64.100.200.0 0.0.0.3 any
 permit ip 64.104.223.0 0.0.0.3 any
 permit icmp 64.100.200.0 0.0.0.3 any
 permit icmp 64.104.223.0 0.0.0.3 any
 permit icmp any any echo-reply
 deny ip any any
banner login Username: administrator
Password: cisco Enable: class
banner motd Username:administrator
Password:cisco Enable:class
line con 0
line aux 0
line vty 0
 login local
 transport input ssh
line vty 1 4
 no login
end

\end{verbatim}
\normalsize
\end{mdframed}

\mysubsection{2}{Filling in part of the table}\\
We now have enough information to fill in pieces of the table

\begin{tabbing}
  interface \= 10.255.255.245 \= 255.255.255.252 \= connecting \= interface\kill
  R2 \> \> \> connecting \> device\\
interface \> address \> subnetmask \> name \> interface\\
G0/0 \> 10.255.255.245 \> 255.255.255.252 \> D1 \> Gi 0/1 \\
G0/1 \> 10.255.255.249 \> 255.255.255.252 \> D2 \> G 0/1 \\
G0/2 \> 10.10.10.1 \> 255.255.255.0 \> S3 \> G 0/1
\end{tabbing}

\newpage

%===================================
\mysection{Part 2: S3}

\mysubsection{1}{Gathering data}\\
I continued to do the same for the rest of the switches and interfaces


\noindent\mysubsection{2}{full table}\\
Here is the completed table
\begin{tabbing}
  interface \= 10.255.255.245 \= 255.255.255.252 \= connecting \= interface\kill
  R2 \> \> \> connecting \> device\\
interface \> address \> subnetmask \> name \> interface\\
G0/0 \> 10.255.255.245 \> 255.255.255.252 \> D1 \> Gi 0/1 \\
G0/1 \> 10.255.255.249 \> 255.255.255.252 \> D2 \> G 0/1 \\
G0/2 \> 10.10.10.1 \> 255.255.255.0 \> S3 \> G 0/1 \\
S0/0/0 \> 64.100.100.1 \> 255.255.255.252 \> Internet \> N/A \\
s0/0/1.1 \> 64.100.200.2 \> 255.255.255.252 \> Intranet \> N/A \\
S3 \>  \> \> \> \\
VLAN 1 \> 10.10.10.254 \> 255.255.255.0 \> N/A \> N/A \\
F0/1 \> N/A \> N/A \> Cent. Srvr \> NIC \\
G0/1 \> N/A \> N/A \> R2 \> G0/2 \\
C. Srvr \\ 
NIC \> 10.10.10.2 \> 255.255.255.0 \> S3 \> F0/1 \\
D1 \> \> \> \> \\
VLAN2 \> 10.2.0.1 \> 255.255.255.0 \> N/A \> N/A\\
G0/1 \> 10.255.255.246 \> 255.255.255.252 \> R2 \> G0/0 \\
G0/2 \> 10.255.255.254 \> 255.255.255.252 \> D2 \> G0/2 \\
F0/23 \> N/A \> N/A \> S2 \> F0/23 \\
F0/24\> N/A \> N/A \> S1 \> G0/1 \\
S1 \> \> \> \> \\
VLAN2 \> 10.2.0.2 \> 255.255.255.0 \> N/A \> N/A \\
F0/23 \> N/A \> N/A \> D2 \> F0/23 \\
G0/1 \> N/A \> N/A \> D1 \> F0/24 \\
D2 \> \> \> \> \\
F0/23 \> N/A \> N/A \> S1 \> F0/23 \\
F0/24 \> 10.3.0.1 \> 255.255.255.0 \> S3 \> G0/1 \\
G0/1 \> 10.255.255.250 \> 255.255.255.252 \> R2 \> G0/1 \\
G0/2 \> 10.255.255.253 \> 255.255.255.252 \> D1 \> G0/2 \\
S2 \> \> \> \> \\
VLAN1 \> 10.3.0.1 \> 255.255.255.0 \> N/A \> N/A \\
F0/23 \> N/A \> N/A \> D1 \> F0/23 \\
G0/1 \> N/A \> N/A \> D2 \> F0/24 
\end{tabbing}

\end{document}

